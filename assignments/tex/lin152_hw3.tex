\documentclass{article}
\usepackage{assignment_preamble}

\title{Homework 3}
\author{Ravi Kini}
\date{February 29, 2024}

\begin{document}

\maketitle

\problem
The distinction between nouns and verbs is a near-abolute universal, but not a universal; some languages, such as Mwotlap, have noun that are marked with tense, mood and aspect, properties typically reserved for verbs.

\clearpage

\problem
\subproblem{(a)}
True; most languages do not mark gender on either nouns or pronouns.
\subproblem{(b)}
True; some languages, like Ju|'hoan, have a grammatical gender system that is not sex-based.
\subproblem{(c)}
True; some languages, like Lak, have a grammatical gender system with a combination of sex-based and non-sex-based marking.
\subproblem{(d)}
True; most languages distinguish between singular and plural in pronoun systems, while this distinction is less common for nouns.

\clearpage

\problem
\subproblem{(a)}
True; languages that grammatically distinguish the dual also grammatically distinguish the plural.
\subproblem{(b)}
False; languages that distinguish plural and dual use mor morphemes to mark the dual.
\subproblem{(c)}
True; as singular is the most common number, it is the least marked.

\clearpage

\problem
(i) is correct, and (ii) is incorrect. As \textit{child} is more animate than \textit{food} and Sesotho has an animacy-based rule of linear ordering, \textit{child} should precede \textit{food} in a grammatical sentence.

\clearpage

\problem
\textit{The round ball} is an example of adjectives being used as modifiers, while \textit{The ball is round} is an example of adjectives being used in predication.

\clearpage

\problem
Languages that lack adjectives express attributes using nouns or verbs. For example, \textit{a happy person} would be expressed as \textit{a person having happiness}.

\clearpage

\problem
Talmy proposes a dichotomy between verb-framed languages like Arabic, which do not express the manner in the verb, and satellite-framed languages like Slavic, which express the manner in the verb. Equipollently-framed languages, like Mandarin, express both path and manner in the verb.

\clearpage

\problem
In Spanish, information about the manner of motion is often optional. As Spanish is verb-framed, manner verbs are few and have lexical and syntactic restrictions on their usage; in this example, the manner verb \textit{running} cannot be combined with the particle of direction \textit{into}.

\clearpage

\problem
Thinking-for-speaking is the idea that language speakers conform their thinking to what the language considers essential to convey. For example, speakers of verb-framed languages tend to express manner less frequently, as it is not necessary.

\clearpage

\problem
Languages can also express the intentionality of motion. For example, Spanish uses different constructions to express that something was dropped intentionally, as opposed to accidentally.

\clearpage

\problem
In Serbian, verbs that have already been prefixed do not allow additional suffixation.

\end{document}