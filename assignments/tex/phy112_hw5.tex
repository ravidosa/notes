\documentclass{article}
\usepackage{assignment_preamble}

\title{Homework 5}
\author{Ravi Kini}
\date{November 9, 2023}

\begin{document}

\maketitle

\problem[\textit{An Introduction to Thermal Physics} (Schroeder, 1e) Exercise 5.12]
If $S$ is held constant in the thermodynamic identity for $U$:
\begin{equation}
    \begin{split}
        \diff{U} & = T \, \diff{S} - p \, \diff{V} = -p \, \diff{V} \\
        {\left(\pderiv{U}{V}\right)}_S & = -p
    \end{split}
\end{equation}
If $V$ is held constant:
\begin{equation}
    \begin{split}
        \diff{U} & = T \, \diff{S} - p \, \diff{V} = T \, \diff{S} \\
        {\left(\pderiv{U}{S}\right)}_V & = T
    \end{split}
\end{equation}
Then:
\begin{equation}
    \begin{split}
        \pderiv{}{V}{\left(\pderiv{U}{S}\right)}_V & = \pderiv{}{S}{\left(\pderiv{U}{V}\right)}_S \\
        {\left(\pderiv{T}{V}\right)}_S & = -{\left(\pderiv{p}{S}\right)}_V
    \end{split}
\end{equation}

\clearpage

\problem[\textit{An Introduction to Thermal Physics} (Schroeder, 1e) Exercise 5.23 (partial)]
\subproblem{(a)}
Deriving the thermodynamic identity for $\Phi$:
\begin{equation}
    \begin{split}
        \Phi & = U - TS - \mu N \\
        \diff{\Phi} & = \diff{U} - \diff{(TS)} - \diff{(\mu N)} \\
        & = \diff{U} - T \, \diff{S} - S \, \diff{T} - \mu \, \diff{N} - N \, \diff{\mu} \\
        & = (T \, \diff{S} - p \, \diff{V} + \mu \, \diff{N}) - T \, \diff{S} - S \, \diff{T} - \mu \, \diff{N} - N \, \diff{\mu} \\
        & = - p \, \diff{V} - S \, \diff{T} - N \, \diff{\mu} \\
    \end{split}
\end{equation}
If $V$ and $T$ are held constant in the thermodynamic identity for $\Phi$:
\begin{equation}
    \begin{split}
        \diff{\Phi} & = - p \, \diff{V} - S \, \diff{T} - N \, \diff{\mu} = - N \, \diff{\mu} \\
        {\left(\pderiv{\Phi}{\mu}\right)}_{V,T} & = -N
    \end{split}
\end{equation}
If $V$ and $\mu$ are held constant:
\begin{equation}
    \begin{split}
        \diff{\Phi} & = - p \, \diff{V} - S \, \diff{T} - N \, \diff{\mu} = - S \, \diff{T} \\
        {\left(\pderiv{\Phi}{\mu}\right)}_{V,\mu} & = -S
    \end{split}
\end{equation}
If $T$ and $\mu$ are held constant:
\begin{equation}
    \begin{split}
        \diff{\Phi} & = - p \, \diff{V} - S \, \diff{T} - N \, \diff{\mu} = - p \, \diff{V} \\
        {\left(\pderiv{\Phi}{\mu}\right)}_{T,\mu} & = -p
    \end{split}
\end{equation}
\subproblem{(b)}
For a system in thermal equilibrium, $T, \mu$ remain constant and equal for the system and the reservoir. Further assume that the volume $V_r$ remains fixed for the reservoir. Let the entropy of the universe (system and reservoir) increase by some $\diff{S}_{u} = \diff{S} + \diff{S}_r$ where $\diff{S}$ is the change in entropy of the system and $\diff{S}_r$ the change in entropy of the reservoir. Then:
\begin{equation}
    \begin{split}
        \diff{U}_r & = T \, \diff{S}_r - P \, \diff{V}_r + \mu \, \diff{N}_r  \\
        \diff{S}_r & = \frac{1}{T}\diff{U}_r - \frac{\mu}{T} \, \diff{N}_r \\
        & = -\frac{1}{T}\diff{U} + \frac{\mu}{T} \, \diff{N} \\
        \diff{S}_u & = \diff{S} -\frac{1}{T}\diff{U} + \frac{\mu}{T} \, \diff{N} \\
        & = -\frac{1}{T}\left(\diff{U} - T \, \diff{S} - \mu \, \diff{N}\right) = -\frac{1}{T}\diff{\Phi}
    \end{split}
\end{equation}
Since the entropy tends to increase spontaneously, $\diff{S}_u$ is positive, and $\diff{\Phi}$ is negative, indicating that the grand free energy $\Phi$ tends to decrease.

\clearpage

\problem[\textit{An Introduction to Thermal Physics} (Schroeder, 1e) Exercise 6.5]
\subproblem{(a)}
The partition function of the particle $Z$ is:
\begin{equation}
    \begin{split}
        Z & = e^{-\frac{-0.05~\unit{\joule}}{k_BT}} + e^{-\frac{0~\unit{\joule}}{k_BT}} + e^{-\frac{0.05~\unit{\joule}}{k_BT}} \\
        & = e^{-\frac{-0.05~\unit{\joule}}{8.617 \cdot 10^{-5}~\unit{\joule\per\kelvin} \cdot 300~\unit{\joule}}} + e^{-\frac{0~\unit{\joule}}{8.617 \cdot 10^{-5}~\unit{\joule\per\kelvin} \cdot 300~\unit{\joule}}} \\
        & + e^{-\frac{0.05~\unit{\joule}}{8.617 \cdot 10^{-5}~\unit{\joule\per\kelvin} \cdot 300~\unit{\joule}}} \approx 8.063 \\
    \end{split}
\end{equation}
\subproblem{(b)}
The probabilities of the particle being in each of the states is:
\begin{equation}
    \begin{split}
        P\left(E = -0.05~\unit{\joule}\right) & = \frac{e^{-\frac{-0.05~\unit{\joule}}{8.617 \cdot 10^{-5}~\unit{\joule\per\kelvin} \cdot 300~\unit{\joule}}}}{Z} \approx 0.858 \\
        P\left(E = 0~\unit{\joule}\right) & = \frac{e^{-\frac{0~\unit{\joule}}{8.617 \cdot 10^{-5}~\unit{\joule\per\kelvin} \cdot 300~\unit{\joule}}}}{Z} \approx 0.124 \\
        P\left(E = 0.05~\unit{\joule}\right) & = \frac{e^{-\frac{0.05~\unit{\joule}}{8.617 \cdot 10^{-5}~\unit{\joule\per\kelvin} \cdot 300~\unit{\joule}}}}{Z} \approx 0.018 \\
    \end{split}
\end{equation}
\subproblem{(c)}
Shifting the zero point and repeating the calculations:
\begin{equation}
    \begin{split}
        Z & = e^{-\frac{0~\unit{\joule}}{k_BT}} + e^{-\frac{0.05~\unit{\joule}}{k_BT}} + e^{-\frac{0.10~\unit{\joule}}{k_BT}} \\
        & = e^{-\frac{0~\unit{\joule}}{8.617 \cdot 10^{-5}~\unit{\joule\per\kelvin} \cdot 300~\unit{\joule}}} + e^{-\frac{0.05~\unit{\joule}}{8.617 \cdot 10^{-5}~\unit{\joule\per\kelvin} \cdot 300~\unit{\joule}}} \\
        & + e^{-\frac{0.10~\unit{\joule}}{8.617 \cdot 10^{-5}~\unit{\joule\per\kelvin} \cdot 300~\unit{\joule}}} \approx 1.165 \\
        P\left(E = 0~\unit{\joule}\right) & = \frac{e^{-\frac{0~\unit{\joule}}{8.617 \cdot 10^{-5}~\unit{\joule\per\kelvin} \cdot 300~\unit{\joule}}}}{Z} \approx 0.858 \\
        P\left(E = 0.05~\unit{\joule}\right) & = \frac{e^{-\frac{0.05~\unit{\joule}}{8.617 \cdot 10^{-5}~\unit{\joule\per\kelvin} \cdot 300~\unit{\joule}}}}{Z} \approx 0.124 \\
        P\left(E = 0.10~\unit{\joule}\right) & = \frac{e^{-\frac{0.10~\unit{\joule}}{8.617 \cdot 10^{-5}~\unit{\joule\per\kelvin} \cdot 300~\unit{\joule}}}}{Z} \approx 0.018 \\
    \end{split}
\end{equation}
The partition function changes, but the probabilities remain the same, since the zero for measuring energy is arbitrary.

\clearpage

\problem[\textit{An Introduction to Thermal Physics} (Schroeder, 1e) Exercise 6.11]
The partition function of a lithium nucleus is:
\begin{equation}
    \begin{split}
        Z & = e^{-\frac{-\frac{3}{2}\mu B}{k_BT}} + e^{-\frac{-\frac{1}{2}\mu B}{k_BT}} + e^{-\frac{\frac{1}{2}\mu B}{k_BT}} + e^{-\frac{\frac{3}{2}\mu B}{k_BT}} \\
        & = e^{-\frac{-\frac{3}{2} \cdot 1.03 \cdot 10^{-7}~\unit{eV\per\tesla}\cdot 0.63~\unit{\tesla}}{8.617 \cdot 10^{-5}~\unit{eV\per\kelvin} \cdot 300~\unit{\kelvin}}} + e^{-\frac{-\frac{1}{2} \cdot 1.03 \cdot 10^{-7}~\unit{eV\per\tesla}\cdot 0.63~\unit{\tesla}}{8.617 \cdot 10^{-5}~\unit{eV\per\kelvin} \cdot 300~\unit{\kelvin}}} \\
        & + e^{-\frac{\frac{1}{2} \cdot 1.03 \cdot 10^{-7}~\unit{eV\per\tesla}\cdot 0.63~\unit{\tesla}}{8.617 \cdot 10^{-5}~\unit{eV\per\kelvin} \cdot 300~\unit{\kelvin}}} + e^{-\frac{\frac{3}{2} \cdot 1.03 \cdot 10^{-7}~\unit{eV\per\tesla}\cdot 0.63~\unit{\tesla}}{8.617 \cdot 10^{-5}~\unit{eV\per\kelvin} \cdot 300~\unit{\kelvin}}} \approx 4 \\
    \end{split}
\end{equation}
The probability of the nucleus being in each of the spin orientations is then:
\begin{equation}
    \begin{split}
        P\left(m = -\frac{3}{2}\right) & = \frac{e^{-\frac{-\frac{3}{2} \cdot 1.03 \cdot 10^{-7}~\unit{eV\per\tesla}\cdot 0.63~\unit{\tesla}}{8.617 \cdot 10^{-5}~\unit{eV\per\kelvin} \cdot 300~\unit{\kelvin}}}}{Z} \approx 0.25 \\
        P\left(m = -\frac{1}{2}\right) & = \frac{e^{-\frac{-\frac{1}{2} \cdot 1.03 \cdot 10^{-7}~\unit{eV\per\tesla}\cdot 0.63~\unit{\tesla}}{8.617 \cdot 10^{-5}~\unit{eV\per\kelvin} \cdot 300~\unit{\kelvin}}}}{Z} \approx 0.25 \\
        P\left(m = \frac{1}{2}\right) & = \frac{e^{-\frac{\frac{a}{2} \cdot 1.03 \cdot 10^{-7}~\unit{eV\per\tesla}\cdot 0.63~\unit{\tesla}}{8.617 \cdot 10^{-5}~\unit{eV\per\kelvin} \cdot 300~\unit{\kelvin}}}}{Z} \approx 0.25 \\
        P\left(m = \frac{3}{2}\right) & = \frac{e^{-\frac{\frac{3}{2} \cdot 1.03 \cdot 10^{-7}~\unit{eV\per\tesla}\cdot 0.63~\unit{\tesla}}{8.617 \cdot 10^{-5}~\unit{eV\per\kelvin} \cdot 300~\unit{\kelvin}}}}{Z} \approx 0.25 \\
    \end{split}
\end{equation}
If both $B$ and $T$ change sign, $\frac{m\mu B}{k_BT}$ does not change sign, so the expression for the probability remains the same. Consequently, the particles obey the Boltzmann distribution for $T = -300 \unit{\kelvin}$.

\clearpage

\problem[\textit{An Introduction to Thermal Physics} (Schroeder, 1e) Exercise 6.14]
For a single air molecule, with $s_1$ being the state where the molecule is at sea level and $s_2$ being the state where the molecule is at height $z$:
\begin{equation}
    \begin{split}
        E\left(s_1\right) & = E\left(s_2\right) + mgz \\
        \frac{P\left(s = s_1\right)}{P\left(s = s_2\right)} & = \frac{e^{-\frac{E\left(s_1\right)}{k_BT}}}{e^{-\frac{E\left(s_2\right)}{k_BT}}} = e^{-\frac{E\left(s_1\right) - E\left(s_2\right)}{k_BT}} = e^{-\frac{mgz}{k_BT}} \\
        P\left(s = s_1\right) & = P\left(s = s_2\right)e^{-\frac{mgz}{k_BT}} \\
        P\left(z\right) & = P\left(0\right)e^{-\frac{mgz}{k_BT}} \\
        \rho\left(z\right) & = \rho\left(0\right)e^{-\frac{mgz}{k_BT}} \\
    \end{split}
\end{equation}

\clearpage

\problem[\textit{An Introduction to Thermal Physics} (Schroeder, 1e) Exercise 6.16]
For a system in equilibrium with a reservoir at temperature $T$, where $\beta = \frac{1}{k_BT}$:
\begin{equation}
    \begin{split}
        -\deriv{}{\beta} \ln{Z} & = -\frac{\deriv{Z}{\beta}}{Z} = -\frac{\deriv{}{\beta}\sum_s e^{-\beta E\left(s\right)}}{Z} \\
        & = -\frac{\sum_s \deriv{}{\beta} e^{-\beta E\left(s\right)}}{Z} \\
        & = -\frac{\sum_s -E\left(s\right)e^{-\beta E\left(s\right)}}{Z} = \frac{\sum_s E\left(s\right)e^{-\beta E\left(s\right)}}{Z} = \overline{E}
    \end{split}
\end{equation}

\clearpage

\problem[\textit{An Introduction to Thermal Physics} (Schroeder, 1e) Exercise 6.20 (partial)]
\subproblem{(a)}
The partition function of a single harmonic oscillator $Z$ is:
\begin{equation}
    \begin{split}
        Z & = \sum_s e^{-E\left(s\right)\beta} \\
        & = \sum_{n=0}^\infty e^{-nhf\beta} \\
        & = \sum_{n=0}^\infty {\left(e^{-hf\beta}\right)}^n \\
        & = \frac{1}{1 - e^{-hf\beta}}
    \end{split}
\end{equation}
\subproblem{(b)}
At temperature $T$, the average energy of a single harmonic oscillator $\overline{E}$ is:
\begin{equation}
    \begin{split}
        \overline{E} & = -\deriv{}{\beta} \ln{Z} \\
        & = -\deriv{}{\beta} \ln{\left(\frac{1}{1 - e^{-hf\beta}}\right)} \\
        & = \deriv{}{\beta} \ln{\left(1 - e^{-hf\beta}\right)} \\
        & = \frac{hfe^{-hf\beta}}{1 - e^{-hf\beta}} \\
        & = \frac{hf}{e^{hf\beta} - 1} = \frac{hf}{e^{\frac{hf}{k_BT}} - 1} \\
    \end{split}
\end{equation}
\subproblem{(c)}
The total energy of a system with $N$ oscillators $U$ is then:
\begin{equation}
    \begin{split}
        U & = N\overline{E} \\
        & = \frac{Nhf}{e^{\frac{hf}{k_BT}} - 1}
    \end{split}
\end{equation}
\subproblem{(d)}
The heat capacity of this system $C$ is:
\begin{equation}
    \begin{split}
        C & = \pderiv{U}{T} = -Nhf\frac{-\frac{hf}{k_B T^2}e^{\frac{hf}{k_BT}}}{{\left(e^{\frac{hf}{k_BT}} - 1\right)}^2} \\
        & = \frac{Nh^2f^2}{k_BT^2}\frac{e^{\frac{hf}{k_BT}}}{{\left(e^{\frac{hf}{k_BT}} - 1\right)}^2} \\
    \end{split}
\end{equation}
In the low-temperature limit:
\begin{equation}
    \begin{split}
        \lim_{T \to 0} C & = \lim_{T \to 0} \frac{Nh^2f^2}{k_BT^2}\frac{e^{\frac{hf}{k_BT}}}{{\left(e^{\frac{hf}{k_BT}} - 1\right)}^2} = 0 \\
    \end{split}
\end{equation}
In the high-temperature limit:
\begin{equation}
    \begin{split}
        \lim_{T \to \infty} C & = \lim_{T \to \infty} \frac{Nh^2f^2}{k_BT^2}\frac{e^{\frac{hf}{k_BT}}}{{\left(e^{\frac{hf}{k_BT}} - 1\right)}^2} \\
        & = \lim_{T \to \infty} \frac{Nh^2f^2}{k_BT^2}\frac{1 + \frac{hf}{k_BT} + \ldots }{{\left(\frac{hf}{k_BT} + \frac{1}{2}{\left(\frac{hf}{k_BT}\right)}^2 + \ldots \right)}^2} \\
        & = \lim_{T \to \infty} \frac{Nh^2f^2}{k_BT^2}\frac{1 + \ldots}{{\left(\frac{hf}{k_BT}\right)}^2 + \ldots} \\
        & = Nk_B \\
    \end{split}
\end{equation}

\clearpage

\problem[\textit{An Introduction to Thermal Physics} (Schroeder, 1e) Exercise 6.26]
Approximating the rotational partition function $Z_{\text{rot}}$:
\begin{equation}
    \begin{split}
        Z_{\text{rot}} & = \sum_{j=0}^{\infty} (2j + 1)e^{-\frac{j(j + 1)\epsilon}{k_BT}}  \approx 1 + 3e^{-\frac{2\epsilon}{k_BT}} = 1 + 3e^{-2\epsilon\beta} \\
    \end{split}
\end{equation}
Using this approximation to find the average energy $\overline{E}_{\text{rot}}$ and heat capacity $C$:
\begin{equation}
    \begin{split}
        \overline{E}_{\text{rot}} & = -\deriv{}{\beta}\ln Z_{\text{rot}} \approx -\deriv{}{\beta}\ln(1 + 3e^{-2\epsilon\beta}) = \frac{6\epsilon e^{-2\epsilon\beta}}{1 + 3e^{-2\epsilon\beta}} \\
        & \approx 6\epsilon e^{-2\epsilon\beta} = 6\epsilon e^{-\frac{2\epsilon}{k_BT}} \\
        C & = \pderiv{\overline{E}}{T} \approx \pderiv{}{T} 6\epsilon e^{-\frac{2\epsilon}{k_BT}} \\
        & \approx 6\epsilon \frac{2\epsilon}{k_BT^2}e^{-\frac{2\epsilon}{k_BT}} = \frac{12\epsilon^2}{k_BT^2}e^{-\frac{2\epsilon}{k_BT}} \\
        \lim_{T\to 0} c & = \lim_{T \to 0} \frac{12\epsilon^2}{k_BT^2}e^{-\frac{2\epsilon}{k_BT}} = 0
    \end{split}
\end{equation}
This result is consistent with the third law, as the heat capacity goes to $0$ as the temperature goes to $0$.

\begin{tikzpicture}
\begin{axis}[samples=200,xlabel=$T$,ylabel=$C$,ytick={0,1},yticklabels={$0$,$1$},xtick={},xmin=0,xmax=2]
\addplot[color=red,domain=0:0.5]{1 / x^2 * e^(-2 / x)};
\addplot[color=red,domain=0.5:1]{(1 - 0.0732625556)/0.5 * (x - 0.5) + 0.0732625556};
\addplot[color=red,domain=1:2]{x/x};
\end{axis}
\end{tikzpicture}

\end{document}