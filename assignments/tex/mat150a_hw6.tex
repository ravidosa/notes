\documentclass{article}
\usepackage{assignment_preamble}

\title{Homework 6}
\author{Ravi Kini}
\date{November 21, 2023}

\begin{document}

\maketitle

\problem[\textit{Algebra} (Artin, 2e) Exercise 6.3.4 (extended)]
A glide reflection can be written as $g = \left(t_a\rho_{\phi}\right)t_vr\left(t_a\rho_{\phi}\right)^{-1}$ where $a, \phi$ are determined by the glide line and $v$ is the glide vector. For all $m \in \Isom\left(\R^2\right)$, $m = t_m\rho_{\theta}$ or $t_m\rho_{\theta}r$, for a unique vector $a$ and angle $\theta$. Suppose $m = t_m\rho_{\theta}$. Then:
\begin{equation}
    \begin{split}
        mgm^{-1} & = \left(t_m\rho_{\theta}\right)\left(t_a\rho_{\phi}\right)t_vr\left(t_a\rho_{\phi}\right)^{-1}\left(t_m\rho_{\theta}\right)^{-1} \\
        & = \left(t_m\rho_{\theta}t_a\rho_{\phi}\right)t_vr\left(t_m\rho_{\theta}t_a\rho_{\phi}\right)^{-1} \\
        & = \left(t_mt_{\rho_{\theta}\left(a\right)}\rho_{\theta}\rho_{\phi}\right)t_vr\left(t_mt_{\rho_{\theta}\left(a\right)}\rho_{\theta}\rho_{\phi}\right)^{-1} \\
        & = \left(t_{m + \rho_{\theta}\left(a\right)}\rho_{\theta + \phi}\right)t_vr\left(t_{m + \rho_{\theta}\left(a\right)}\rho_{\theta + \phi}\right)^{-1} \\
        & = \left(t_{a'}\rho_{\phi'}\right)t_vr\left(t_{a'}\rho_{\phi'}\right)^{-1} \\
    \end{split}
\end{equation}
Evidently, the conjugate is also a glide vector where $a' = m + \rho_{\theta}\left(a\right), \phi' = \theta + \phi$. As the glide vector is the same glide vector, they naturally have the same length. Now suppose $m = t_m\rho_{\theta}r$. Then:
\begin{equation}
    \begin{split}
        mgm^{-1} & = \left(t_m\rho_{\theta}r\right)\left(t_a\rho_{\phi}\right)t_vr\left(t_a\rho_{\phi}\right)^{-1}\left(t_m\rho_{\theta}r\right)^{-1} \\
        & = \left(t_m\rho_{\theta}rt_a\rho_{\phi}\right)t_vr\left(t_m\rho_{\theta}rt_a\rho_{\phi}\right)^{-1} \\
        & = \left(t_m\rho_{\theta}t_{r\left(a\right)}r\rho_{\phi}\right)t_vr\left(t_m\rho_{\theta}t_{r\left(a\right)}r\rho_{\phi}\right)^{-1} \\
        & = \left(t_m\rho_{\theta}t_{r\left(a\right)}\rho_{-\phi}r\right)t_vr\left(t_m\rho_{\theta}t_{r\left(a\right)}\rho_{-\phi}r\right)^{-1} \\
        & = \left(t_m\rho_{\theta}t_{r\left(a\right)}\rho_{-\phi}\right)rt_vrr\left(t_m\rho_{\theta}t_{r\left(a\right)}\rho_{-\phi}\right)^{-1} \\
        & = \left(t_mt_{\rho_{\theta}\left(r\left(a\right)\right)}\rho_{\theta}\rho_{-\phi}\right)rt_v\left(t_mt_{\rho_{\theta}\left(r\left(a\right)\right)}\rho_{\theta}\rho_{-\phi}\right)^{-1} \\
        & = \left(t_{m + \rho_{\theta}\left(r\left(a\right)\right)}\rho_{\theta - \phi}\right)t_{r\left(v\right)}r\left(t_{m + \rho_{\theta}\left(r\left(a\right)\right)}\rho_{\theta - \phi}\right)^{-1} \\
        & = \left(t_{a'}\rho_{\phi'}\right)t_vr\left(t_{a'}\rho_{\phi'}\right)^{-1} \\
    \end{split}
\end{equation}
Evidently, the conjugate is also a glide vector where $a' = m + \rho_{\theta}\left(r\left(a\right)\right), \phi' = \theta - \phi$. As the glide vector is the original glide vector under an isometry, length is preserved and $v, r\left(v\right)$ have the same length. In both cases, the conjugate of a glide reflection is a glide reflection that has a glide vector with the same length.

\clearpage

\problem
The generators of $\Isom\left(\R^2\right)$ are known to be $t_a, \rho_{\theta}, r$ where:
\begin{equation}
    \begin{split}
        t_a\left(x\right) & = \begin{bmatrix} x_1 \\ x_2\end{bmatrix} + \begin{bmatrix} a_1 \\ a_2\end{bmatrix} \\
        \rho_{\theta}\left(x\right) & = \begin{bmatrix} \cos\theta & -\sin\theta \\ \sin\theta & \cos\theta \end{bmatrix}\begin{bmatrix} x_1 \\ x_2\end{bmatrix} \\
        r\left(x\right) & = \begin{bmatrix} 1 & 0 \\ 0 & -1 \end{bmatrix}\begin{bmatrix} x_1 \\ x_2\end{bmatrix} \\
    \end{split}
\end{equation}
Let $z \in \C$. Translation in $\R^2$ is addition in $\C$. Consequently:
\begin{equation}
    \begin{split}
        t_{z_0}\left(z\right) & = z + z_0 \\
        \left|t_{z_0}\left(z'\right) - t_{z_0}\left(z\right)\right| & = \left|\left(z' + z_0\right) - \left(z + z_0\right)\right| \\
        & = \left|z' - z\right| \\
    \end{split}
\end{equation}
Rotation about the origin in $\R^2$ is multiplication in $\C$ by a member of $S^1$. Consequently:
\begin{equation}
    \begin{split}
        \rho_{\theta}\left(z\right) & = ze^{i\theta} \\
        \left|\rho_{\theta}\left(z'\right) - \rho_{\theta}\left(z\right)\right| & = \left|z'e^{i\theta} - ze^{i\theta}\right| \\
        & = \left|\left(z' - z\right)e^{i\theta}\right| \\
        & = \left|z' - z\right|\left|e^{i\theta}\right| \\
        & = \left|z' - z\right| \\
    \end{split}
\end{equation}
Reflection about the x-axis in $\R^2$ is taking the complex conjugate in $\C$. Consequently:
\begin{equation}
    \begin{split}
        r\left(z\right) = \overline{z} \\
        \left|r\left(z'\right) - r\left(z\right)\right| = \left|r\left(x' + iy'\right) - r\left(x + iy\right)\right| & = \left|\left(x' - iy'\right) - \left(x - iy\right)\right| \\
        & = \left|\left(x' - x\right) - i\left(y' - y\right)\right| \\
        & = \left|\overline{\left(x' - x\right) + i\left(y' - y\right)}\right| \\
        & = \left|\left(x' - x\right) + i\left(y' - y\right)\right| \\
        & = \left|\left(x' + iy'\right) - \left(x + iy\right)\right| \\
        & = \left|z' - z\right|
    \end{split}
\end{equation}
Evidently, all three are isometries of $\C$.

\clearpage

\problem[\textit{Algebra} (Artin, 2e) Exercise 6.6.2]
\subproblem{(a)}
Let $G$ denote the group of symmetries of the following infinite wallpaper pattern constructed from equilateral triangles of side length 1. By inspection, we see that reflection across the $x$-axis and rotation by a multiple of $\frac{2\pi}{6} = \frac{\pi}{3}$ are symmetries of the wallpaper pattern. Conseqquently, the point group $\overline{G} = \left\{\overline{1}, \overline{r}, \overline{\rho_{\frac{\pi}{3}}}, \overline{\rho_{\frac{\pi}{3}}r}\right\} = D_6$. The index of $L$ in $G$ is the the order of $\overline{G} = D_6$, which is $2\left(6\right) = 12$.
\subproblem{(b)}
By inspection, we see that $L$ is the lattice $\Z a + \Z b$ for $a = \begin{bmatrix}
    \frac{1}{2} \\ \frac{\sqrt{3}}{2}
\end{bmatrix}, b = \begin{bmatrix}
    -\frac{1}{2} \\ \frac{\sqrt{3}}{2}
\end{bmatrix}$.

\end{document}