\documentclass{article}
\usepackage{assignment_preamble}

\title{Homework 2}
\author{Ravi Kini}
\date{February 1, 2024}

\begin{document}

\maketitle

\problem
Languages with rigid word order (like English) have constraints on the order of constituents in an utterance, while languages with more flexible word order (like Greek) have less such constraints.

\clearpage

\problem
French is prepositional, while Japanese is postpositional.

\clearpage

\problem
\subproblem{(i)}
Relative clauses in japanese are prenominal.
\subproblem{(ii)}
The subject is being relativized.

\clearpage

\problem
Lithuanian marks tense grammatically, using affixes to indicate whether an event took place in the past, present, or future, while Mandarin does not mark tense grammatically, and instead indicates the time of events by explicitly stating the time at which the event took place.

\clearpage

\problem
In agreement, several words are all marked with some semantic feature. For example, in French, \textit{les roses rouges} has all the words marked for both gender and number. In government, one word is not marked, and all those that pair with it are. For example, in English \textit{small dogs} has only the noun marked for number.

\clearpage

\problem
The Accessibility Hierarchy for Relative Clauses indicates the relative computational complexity of relativization for different types of nominals, with relativization of subjects being the least complex, followed by direct objects, indirect objects, oblique complements, and genitives in order of increasing complexity.

\clearpage

\problem

[A \& S][P] is nominative-accusative, while [A][S \& P] is ergative-absolutive.

\clearpage

\problem
The aspect is marked lexically in English, as indicated by the use of \textit{the}, and grammatically in Serbian, as indicated by the information carried by the morphemes.

\clearpage

\problem
Languages tend to have the subject preceding the object (subject saliency) and the object contiguous to the verb (phrase structure rules). SOV and SVO are the two word orders that satisfy both of these principles.

\clearpage

\problem
Languages use case-marking to identify nominals and encode semantic and grammatical information. Languages without case marking indicate grammatical relations through word order, which is then likely rigid.

\end{document}