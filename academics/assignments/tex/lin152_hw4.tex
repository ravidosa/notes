\documentclass{article}
\usepackage{assignment_preamble}

\title{Homework 4}
\author{Ravi Kini}
\date{March 14, 2024}

\begin{document}

\maketitle

\problem
In the context of language change, destruction entails a shift towards more economy, creation entails a shift towards more expressivity, and analogy entails a shift towards more regularity.

\clearpage

\problem
Languages with lower prestige tend to borrow from those with higer prestige, and languages with fewer speaker tend to borrow from those with more speakers.

\clearpage

\problem
Languages that completely overlap in a lexical or grammatical feature are unlikely to change, as they already share a feature, while languages that have no overlap face communicative difficulty that hinders language change. Languages with partial overlap then hav the greatest potential for language change.

\clearpage

\problem
English lost its morphology due to contact with Scandinavian languages, while evidentiality was introduced to Bulgarian due to contact with Turkish.

\clearpage

\problem
The principles of CASP are:
\begin{itemize}
    \item minimizing learning effort: if the two languages share a lexical or grammatical property, a bilingual speaker is able to minimize the effort needed to learn.
    \item minimizing processing effort: bilingual speakers prefer simple, more quickly processed structures to more complex ones.
    \item maximizing expressive power: bilingual speakers seek to express themselves equally well in the two languages.
    \item maximizing efficiency of communication: bilingual speakers use simple constructions when possible and more complex constructions when necessary.
    \item maximizing common ground: bilingual speakers use shared properties of the two languages when they overlap, and may introduce structures from one language to the other when no such overlap exists or avoid using structures that do not overlap.
\end{itemize}

\clearpage

\problem
In French and Spanish, the adjective tends to follow the noun, with the reverse occuring infrequently. However, in bilingual contact with English, where the noun follows the adjective, the frequency of the latter increases.

\clearpage

\problem
Language contact with Scandinavian languages led to English losing its morphology. Bilingual English-Russian speakers tend to use overt subjects in Russian despite it being pro-drop. Bilingual English-Spanish speakers tend to either express intentionality in both or neither of their two languages.

\clearpage

\problem
Bilinguals with unbalanced proficiency in their languages seek to increase the common grounds by introducing structures from their L1 to their L2; for example, native Japanese speakers learning English may often drop the definite article, as it is not present in Japanese.

\clearpage

\problem
When a bilingual speaker interacts with a monolingual speaker, less maximization of common ground occurs.

\clearpage

\problem
Although Quechua Spanish held more prestige in comparison to Quechua, due to the high numbers of bilingual speakers seeking to maximize common ground between both of their languages, evidentiality was introduced to Quechua Spanish.

\clearpage

\problem
Maximization of common ground has to be done within the constraints of existing grammatical conventions; languages that are typologically similar are then easier to borrow features from, particularly basic ones like articles.

\clearpage

\problem
English and Spanish differ in their expression of manner; consequently, Spanish and English speakers may judge events as being more violent based on the difference in how the two languages describe the events.

\end{document}