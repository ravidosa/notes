\documentclass{article}
\usepackage{assignment_preamble}

\title{Homework 1}
\author{Ravi Kini}
\date{January 25, 2024}

\begin{document}

\maketitle

\problem
A typology, and by extension, a typological indicator are significant if they indicate more about the languages than solely the parameter relevant to the indicator, with the languages having more in common beyond the shared parameter.

\clearpage

\problem
Absolute universals make a statement that all languages share some property, while non-absolute universals make a statement that languages generally, but not always share some property. Implicational universals make a statement that languages that have a certain property also have a second property, while non-implicational universals have no such connection between properties.

\textit{All languages have vowels} is an absolute non-implicational universal. \textit{Most languages have nasal consonants} is a non-absolute non-implicational universal. \textit{If a language has a first or second person reflexive pronoun, then it has a third person reflexive pronoun} is an absolute implicational universal. \textit{If a language has SOV word order, it is probably postpositional} is a non-absolute implicational universal.

\clearpage

\problem
Genetic classification of languages is done based on a shared origin/parent language, while areal classification of languages is done based on a shared geographical location.

\clearpage

\problem
The four major language families in Africa are Afro-Asiatic, Nilo-Saharan, Niger-Kordofanian, and Khoisan.

\clearpage

\problem
The WALS classifications of the following languages are:
\begin{center}
    \begin{tabular}{|c|c|}
        \hline
        Language & WALS \\
        \hline
        Basque & isolate \\
        \hline
        Berber & Afro-Asiatic : Berber \\
        \hline
        Burmese & Sino-Tibetan : Tibeto-Burman : Burmese-Lolo \\
        \hline
        Burushaski & isolate \\
        \hline
        Chibcha & Chibchan : Chibcha-Duit \\
        \hline
        Finnish & Uralic : Finno-Ugric : Finnic \\
        \hline
        Fula(ni) & Niger-Congo : Atlantic : Peul-Serer \\
        \hline
        Greek & Indo-European : Greek \\
        \hline
        Guarani & Tupi : Tupi-Guarani \\
        \hline
        Hebrew & Afro-Asiatic : Semitic \\
        \hline
        Hindi & Indo-European : Indic \\
        \hline
        Italian & Indo-European : Romance\\
        \hline
        Japanese & Japanese \\
        \hline
        Kannada & Dravidian \\
        \hline
        Loritja & Pama-Nyungan : Western \\
        \hline
        Malay & Austronesian : Malayo-Sumbawan \\
        \hline
        Maori & Austronesian : Eastern Malayo-Polynesian : Oceanic \\
        \hline
        Masai & Eastern Sudanic : Nilotic \\
        \hline
        Maya & Mayan \\
        \hline
        Norwegian & Indo-European : Germanic \\
        \hline
    \end{tabular}
\end{center}

\clearpage

\problem
In fusional languages, it is difficult to segment morphemes, and a lot of semantic/grammatical information is carried by a morpheme, whereas in agglutinative languages, morphemes are easily segmented, with each segment carrying some information. For example, Latin is fusional, while Turkish is agglutinative.

\clearpage

\problem
As defined by Sapir, the index of synthesis indicates the amount of affixation in a language while the index of fusion indicates the ease with which morphemes are segmented. Along the axis of synthesis, isolating languages (like Mandarin) are monomorphemic, while synthetic languages (like Inuktitut) can have utterances composed only of a root with affixed morphemes. Along the axis of fusion, agglutinative languages (like Turkish) are easily segmented while fusional languages (like Spanish) have affixes that carry several pieces of semantic and grammatical information.

\clearpage

\problem
Head marking is present in a language when in a head-dependent construction, the head is marked to indicate syntactic dependence, while in dependent marking, the dependent is marked. Double marking, where both the head and dependent are marked, and split marking, where both head and dependent marking occur in roughly equal proportion, are also language grouping made using this marking criterion.

\clearpage

\problem
Spanish is fusional, as single morphemes carry several pieces of semantic and grammatical function (mode, person, number, tense, aspect), while Lithuanian is agglutinative, as morphemes are easily segmented.

\end{document}