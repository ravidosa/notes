\documentclass{article}
\usepackage{assignment_preamble}

\title{HW 5}
\author{Ravi Kini}
\date{November 7, 2023}

\begin{document}

\maketitle

\problem
Let $G$ be a group generated by some $x, y \in G$, let $N \normal G$, and let $\bar G = G/N$. Let $g \in G$. Since $G$ is generated by $x, y \in G$, for some $g_1, g_2, \ldots g_n \in \left\{x, y, x^{-1}, y^{-1}\right\}$, $g = g_1g_2\ldots g_n$. Consequently, for $gN \in \overline{G}$:
\begin{equation}
    \begin{split}
        gN & = \left(g_1g_2\ldots g_n\right)N \\
        & = \left(g_1N\right)\left(g_2N\right)\ldots\left(g_nN\right) \\
        & = \overline{g}_1\overline{g}_2\ldots\overline{g}_n
    \end{split}
\end{equation}
Evidently, every element in $\overline{G}$ can be expressed as $\overline{g}_1\overline{g}_2\ldots\overline{g}_n$ for some $\overline{g}_1, \overline{g}_2\ldots \overline{g}_n \in \left\{\overline{x}, \overline{y}, \overline{x^{-1}}, \overline{y^{-1}}\right\}$. Therefore $\overline{G}$ is generated by $\overline{x}, \overline{y} \in \overline{G}$.

\newpage

\problem[\textit{Algebra} (Artin, 2e) Exercise 2.12.2 (partial)]
\subsection*{Part (a)}
Evidently, the identity element is in $H$, as it is the element of $H$ where $a, b, c = 0$. We then compute the product of two arbitrary elements of $H$:
\begin{equation}
    \begin{split}
        \begin{bmatrix}
            1 & a & b \\
            0 & 1 & c \\
            0 & 0 & 1 
        \end{bmatrix}
        \begin{bmatrix}
            1 & x & y \\
            0 & 1 & z \\
            0 & 0 & 1 
        \end{bmatrix}
        & = 
        \begin{bmatrix}
            1 & x + a & y + az + b \\
            0 & 1 & z + c \\
            0 & 0 & 1 \\
        \end{bmatrix} = 
        \begin{bmatrix}
            1 & a' & b' \\
            0 & 1 & c' \\
            0 & 0 & 1 \\
        \end{bmatrix}\in H
    \end{split}
\end{equation}
From the above computation, the group exhibits closure, as the product is the element of $H$ where $a' = x + a, b' = y + az + b, c' = z + c$. From the above computation, the group has inverses, as when $x = -a, y = ac - b, z = -c$, the product is the identity. Since $H$ satisfies the identity, closure, and inverse properties, $H \leq GL_3\left(\mathbb{F}\right)$.
\subsection*{Part (b)}
We compute the following product:
\begin{equation}
    \begin{split}
        \begin{bmatrix}
            1 & a & b \\
            0 & 1 & c \\
            0 & 0 & 1 
        \end{bmatrix}
        \begin{bmatrix}
            1 & 0 & y \\
            0 & 1 & 0 \\
            0 & 0 & 1 
        \end{bmatrix}
        \begin{bmatrix}
            1 & a & b \\
            0 & 1 & c \\
            0 & 0 & 1 
        \end{bmatrix}^{-1}
        & = \begin{bmatrix}
            1 & a & b \\
            0 & 1 & c \\
            0 & 0 & 1 
        \end{bmatrix}
        \begin{bmatrix}
            1 & 0 & y \\
            0 & 1 & 0 \\
            0 & 0 & 1 
        \end{bmatrix}
        \begin{bmatrix}
            1 & -a & ac - b \\
            0 & 1 & -c \\
            0 & 0 & 1 
        \end{bmatrix} \\
        & = 
        \begin{bmatrix}
            1 & a & y + b \\
            0 & 1 & c \\
            0 & 0 & 1 \\
        \end{bmatrix}
        \begin{bmatrix}
            1 & -a & ac - b \\
            0 & 1 & -c \\
            0 & 0 & 1 
        \end{bmatrix} \\
        & = 
        \begin{bmatrix}
            1 & 0 & y \\
            0 & 1 & 0 \\
            0 & 0 & 1 \\
        \end{bmatrix}  \in K
    \end{split}
\end{equation}
Evidently, for arbitrary $h \in H$ and $k \in K$, $hkh^{-1} \in K$, so $K \normal H$.
\subsection*{Part (c)}
Let $A = \begin{bmatrix} 1 & a_1 & a_2 \\ 0 & 1 & a_3 \\ 0 & 0 & 1\end{bmatrix}, B = \begin{bmatrix} 1 & b_1 & b_2 \\ 0 & 1 & b_3 \\ 0 & 0 & 1\end{bmatrix} \in H$. From the computation done in (b), we see that $A$ and $B$ are in the same coset of $K$ when $a_1 = b_1$ and $a_3 = b_3$. We can then construct the surjective homomorphism mapping $h = \begin{bmatrix} 1 & h_1 & h_2 \\ 0 & 1 & h_3 \\ 0 & 0 & 1\end{bmatrix} \in H$ to $\overline{h} = \overline{\begin{bmatrix} 1 & h_1 & 0 \\ 0 & 1 & h_3 \\ 0 & 0 & 1\end{bmatrix}} \in \overline{H}$. Evidently, $K$ is the kernel of this homomorphism, as it maps to the identity matrix.

\newpage

\problem
\subsection*{Part (a)}
Let the Klein four group $V = \left\{1, a, b, ab\right\} = \langle a,b \st a^2 = b^2 = \left[a,b\right] = 1\rangle \cong C_2 \times C_2$. Let the subgroup $N = \left\{ e, \left(1~2\right)\left(3~4\right), \left(1~3\right)\left(2~4\right), \left(1~4\right)\left(2~3\right)\right\}$ in $S_4$. We construct the bijective mapping $\varphi : V \to N$ where $1 \to e$, $a \to \left(1~2\right)\left(3~4\right)$, $b \to \left(1~3\right)\left(2~4\right)$, $ab \to \left(1~4\right)\left(2~3\right)$. To show that this is a homomorphism, we construct Cayley tables:
\begin{center}
\begin{tabular}{|c||c|c|c|c|}
    \hline
     & $1$ & $a$ & $b$ & $ab$ \\
    \hline
    \hline
    $1$ & $1$ & $a$ & $b$ & $ab$ \\
    \hline
    $a$ & $a$ & $1$ & $ab$ & $b$ \\
    \hline
    $b$ & $b$ & $ab$ & $1$ & $a$ \\
    \hline
    $ab$ & $ab$ & $b$ & $a$ & $1$ \\
    \hline
\end{tabular}


\vspace{0.25in}

\begin{tabular}{|c||c|c|c|c|}
    \hline
     & $e$ & $\left(1~2\right)\left(3~4\right)$ & $\left(1~3\right)\left(2~4\right)$ & $\left(1~4\right)\left(2~3\right)$ \\
    \hline
    \hline
    $e$ & $e$ & $\left(1~2\right)\left(3~4\right)$ & $\left(1~3\right)\left(2~4\right)$ & $\left(1~4\right)\left(2~3\right)$ \\
    \hline
    $\left(1~2\right)\left(3~4\right)$ & $\left(1~2\right)\left(3~4\right)$ & $e$ & $\left(1~4\right)\left(2~3\right)$ & $\left(1~3\right)\left(2~4\right)$ \\
    \hline
    $\left(1~3\right)\left(2~4\right)$ & $\left(1~3\right)\left(2~4\right)$ & $\left(1~4\right)\left(2~3\right)$ & $e$ & $\left(1~2\right)\left(3~4\right)$ \\
    \hline
    $\left(1~4\right)\left(2~3\right)$ & $\left(1~4\right)\left(2~3\right)$ & $\left(1~3\right)\left(2~4\right)$ & $\left(1~2\right)\left(3~4\right)$ & $e$ \\
    \hline
\end{tabular}
\end{center}
By inspecting the Cayley tables, we see that $\varphi\left(x\right)\varphi\left(y\right) = \varphi\left(xy\right)$ for all $x, y \in V$. As $\varphi$ is a bijective homomorphism, it is an isomorphism and as an isomorphism exists between $V$ and $N$, $N$ is isomorphic to the Klein four group.
\subsection*{Part (b)}
The subgroup $N$ is composed of 1 permutation of cycle type $1,1,1,1$ and permutations of cycle type $2,2$. We note that $e$ is the only permutation of cycle type $1,1,1,1$, as $\frac{\frac{P_1^4}{1}\frac{P_1^3}{1}\frac{P_1^1}{1}\frac{P_1^1}{1}}{4!} = 1$, and that $\left(1~2\right)\left(3~4\right), \left(1~3\right)\left(2~4\right), \left(1~4\right)\left(2~3\right)$ are the only permutations of cycle type $2,2$, as $\frac{\frac{P_2^4}{2}\frac{P_2^2}{2}}{2!} = 3$. As conjugation preserves cycle type, $e$ must be mapped back to itself, the only permutation of its cycle type, and $\left(1~2\right)\left(3~4\right), \left(1~3\right)\left(2~4\right), \left(1~4\right)\left(2~3\right)$ must be mapped to cycles of type $2,2,2$, all of which are elements of $V$. Since for all $p \in V$ and all $q \in S_4$, $qpq^{-1} \in V$, $V \normal S_4$.
\subsection*{Part (c)}
Let $H = \langle \left(1~2\right) , \left(3~4\right) \rangle  \leq S_4$. We note that $H = \langle \left(1~2\right) , \left(3~4\right) \rangle = \left\{e, \left(1~2\right), \left(3~4\right), \left(1~2\right)\left(3~4\right)\right\}$. We construct the bijective mapping $\varphi : V \to H$ where $1 \to e$, $a \to \left(1~2\right)$, $b \to \left(3~4\right)$, $ab \to \left(1~2\right)\left(3~4\right)$. To show that this is a homomorphism, we construct Cayley tables:
\begin{center}
\begin{tabular}{|c||c|c|c|c|}
    \hline
     & 1 & a & b & ab \\
    \hline
    \hline
    1 & 1 & a & b & ab \\
    \hline
    a & a & 1 & ab & b \\
    \hline
    b & b & ab & 1 & a \\
    \hline
    ab & ab & b & a & 1 \\
    \hline
\end{tabular}

\vspace{0.25in}

\begin{tabular}{|c||c|c|c|c|}
    \hline
     & $e$ & $\left(1~2\right)$ & $\left(3~4\right)$ & $\left(1~2\right)\left(3~4\right)$ \\
    \hline
    \hline
    $e$ & $e$ & $\left(1~2\right)$ & $\left(3~4\right)$ & $\left(1~2\right)\left(3~4\right)$ \\
    \hline
    $\left(1~2\right)$ & $\left(1~2\right)$ & $e$ & $\left(1~2\right)\left(3~4\right)$ & $\left(1~2\right)$ \\
    \hline
    $\left(3~4\right)$ & $\left(3~4\right)$ & $\left(1~2\right)\left(3~4\right)$ & $e$ & $\left(3~4\right)$ \\
    \hline
    $\left(1~2\right)\left(3~4\right)$ & $\left(1~2\right)\left(3~4\right)$ & $\left(3~4\right)$ & $\left(1~2\right)$ & $e$ \\
    \hline
\end{tabular}
\end{center}
By inspecting the Cayley tables, we see that $\varphi\left(x\right)\varphi\left(y\right) = \varphi\left(xy\right)$ for all $x, y \in V$. As $\varphi$ is a bijective homomorphism, it is an isomorphism and as an isomorphism exists between $V$ and $H$, $H$ is isomorphic to the Klein four group.

However, $H$ is not normal in $S_4$. To see this, conjugate $\left(1~2\right)$ by $\left(1~2~3~4\right)$. Then:
\begin{equation}
    \begin{split}
        \left(1~2~3~4\right)\left(1~2\right)\left(1~2~3~4\right)^{-1} & = \left(1~2~3~4\right)\left(1~2\right)\left(4~3~2~1\right) \\
        & = \left(1~2~3~4\right)\left(1~4~3\right) \\
        & = \left(2~3\right)
    \end{split}
\end{equation}
Since $\left(2~3\right) \not\in H$, evidently $H$ is not normal in $S_4$.
\subsection*{Part (d)}
We compute the cosets of $N$ in $S_4$.
\begin{equation*}
    \begin{split}
        eN = N \quad & \quad \left(1~2\right)\left(3~4\right)N = N \\ 
        \left(1~3\right)\left(2~4\right)N= N \quad & \quad \left(1~4\right)\left(2~3\right)N = N \\ \\
        \left(1~2\right)N = \left\{\left(1~2\right), \left(3~4\right), \left(1~4~2~3\right), \left(1~3~2~4\right)\right\} \quad & \quad \left(3~4\right)N = \left\{\left(3~4\right), \left(1~2\right), \left(1~4~2~3\right), \left(1~3~2~4\right)\right\} \\
        \left(1~3~2~4\right)N = \left\{\left(1~3~2~4\right), \left(1~4~2~3\right), \left(3~4\right), \left(1~2\right)\right\} \quad & \quad \left(1~4~2~3\right)N = \left\{\left(1~4~2~3\right), \left(1~3~2~4\right), \left(1~2\right), \left(3~4\right)\right\} \\ \\
        \left(2~3\right)N = \left\{\left(2~3\right), \left(1~3~4~2\right), \left(1~2~4~3\right), \left(1~4\right)\right\} \quad & \quad \left(1~3~4~2\right)N = \left\{\left(1~3~4~2\right), \left(2~3\right), \left(1~4\right), \left(1~2~4~3\right)\right\} \\
        \left(1~2~4~3\right)N = \left\{\left(1~2~4~3\right), \left(1~4\right), \left(2~3\right), \left(1~3~4~2\right)\right\} \quad & \quad \left(1~4\right)N = \left\{\left(1~4\right), \left(1~2~4~3\right), \left(1~3~4~2\right), \left(2~3\right)\right\} \\ \\
        \quad \left(1~3~2\right)N = \left\{\left(1~3~2\right), \left(2~3~4\right), \left(1~2~4\right), \left(1~4~3\right)\right\} \quad & \quad \left(2~3~4\right)N = \left\{\left(2~3~4\right), \left(1~3~2\right), \left(1~4~3\right), \left(1~2~4\right)\right\} \\
        \quad \left(1~4~3\right)N = \left\{\left(1~4~3\right), \left(1~2~4\right), \left(2~3~4\right), \left(1~3~2\right)\right\} \quad & \quad \left(1~2~4\right)N = \left\{\left(1~2~4\right), \left(1~4~3\right), \left(1~3~2\right), \left(2~3~4\right)\right\} \\ \\
        \left(2~4~3\right)N = \left\{\left(2~4~3\right), \left(1~4~2\right), \left(1~2~3\right), \left(1~3~4\right)\right\} \quad & \quad \left(1~4~2\right)N = \left\{\left(1~4~2\right), \left(2~4~3\right), \left(1~3~4\right), \left(1~2~3\right)\right\} \\
        \left(1~2~3\right)N = \left\{\left(1~2~3\right), \left(1~3~4\right), \left(2~4~3\right), \left(1~4~2\right)\right\} \quad & \quad \left(1~3~4\right)N = \left\{\left(1~3~4\right), \left(1~2~3\right), \left(1~4~2\right), \left(2~4~3\right)\right\} \\ \\
        \left(1~3\right)N = \left\{\left(1~3\right), \left(1~2~3~4\right), \left(2~4\right), \left(1~4~3~2\right)\right\} \quad & \quad \left(1~4~3~2\right)N = \left\{\left(1~4~3~2\right), \left(2~4\right), \left(1~2~3~4\right), \left(1~3\right)\right\} \\
        \left(2~4\right)N = \left\{\left(2~4\right), \left(1~4~3~2\right), \left(1~3\right), \left(1~2~3~4\right)\right\} \quad & \quad \left(1~2~3~4\right)N = \left\{\left(1~2~3~4\right), \left(1~3\right), \left(1~4~3~2\right), \left(2~4\right)\right\} \\
    \end{split}
\end{equation*}
Consequently, we define $S_4 / N = \left\{\overline{e} ,\overline{\left(1~2\right)},  \overline{\left(2~3\right)}, \overline{\left(1~3~2\right)}, \overline{\left(1~2~3\right)}, \overline{\left(1~3\right)}\right\}$, noting that each coset has exactly one element where $4$ permutes to itself. We then define the biijection $\varphi : S_4 / N \to S_3$ that maps $\overline{\sigma} \to \sigma$.  It is easily verified that $\varphi$ is a homomorphism and that $N$ is the kernel of the homomorphism. As $\varphi$ is a bijective homomorphism, $\varphi$ is an isomorphism and as an isomorphism exists between $S_4 / N$ and $S_3$, $S_4 / N \cong S_3$.
\subsection*{Part (e)}
By the Correspondence Theorem, since $\varphi$ is a subjective homomorphism of which $N$ is the kernel, there exists a bijection between the subgroups of $S_4$ containing $N$ and the subgroups of $S_3$. We note that the inverse of an element of $S_3$ (and, in general, $S_n$) can be found by reversing the order of the individual cycles that compose the permutation (e.g. $\left(1~2~3\right)^{-1} = \left(3~2~1\right) = \left(1~3~2\right)$). Consequently, the three permutations of cycle type $2,1$ are their own inverses, and the two permutations of cycle type $3$ are inverses of each other (the identity permutation of cycle type $1,1,1$ is, of course, its own inverse). Further, the composition of two different permutations of cycle type $2,1$ is a permutation of cycle type $3$. We then enumerate the subgroups of $S_3$ as such: the trivial subgroup, three subgroups containing the identity and one of the permutations of cycle type $2,1$, the subgroup containing the identity and both permutations of cycle type $3$, and $S_3$ itself. As there exist $6$ subgroups of $S_3$, there are therefore $6$ subgroups of $S_4$ that contain $N$.

\end{document}