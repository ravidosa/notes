\documentclass{article}
\usepackage{assignment_preamble}

\title{HW 7}
\author{Ravi Kini}
\date{November 28, 2023}

\begin{document}

\maketitle

\problem[\textit{Algebra} (Artin, 2e) Exercise 6.7.1 (partial)]
\subsection*{Part (a)}
Let $G = D_4$ be the dihedral group of symmetries of the square $[-1,1]^2 \subset \R^2$, generated by $\rho = \rho_{\pi/2}$ and $\tau$ reflection across the $e_1$-axis. By inspection, the elements of $D_4$ that fix $v = \left(1, 1\right)$ are $\left\{e, \rho\tau\right\}$. These correspond, respectively, with performing no actions and performing a reflection across the $e-1$ axis followed by a rotation of $\theta = \frac{\pi}{2}$.
\begin{equation}
    \begin{split}
        v & = \begin{bmatrix} 1 \\ 1 \end{bmatrix} \\
        \tau v & = \begin{bmatrix} 1 \\ -1 \end{bmatrix} \\
        \rho\tau v & = \begin{bmatrix} 1 \\ 1 \end{bmatrix} =  v
    \end{split}
\end{equation}
Evidently, the vertex is mapped to itself.
\subsection*{Part (b)}
By inspection, the elements of $D_4$ that fix the top edge $e$ connecting $\left(-1, 1\right)$ and $\left(1, 1\right)$ are $\left\{e, \rho^2\tau\right\}$. These correspond, respectively, with performing no actions and performing a reflection across the $e-1$ axis followed by two rotations of $\theta = \frac{\pi}{2}$.
\begin{equation}
    \begin{split}
        e & = \begin{bmatrix} x \\ 1 \end{bmatrix} \\
        \tau e & = \begin{bmatrix} x \\ -1 \end{bmatrix} \\
        \rho\tau e & = \begin{bmatrix} 1 \\ x \end{bmatrix} \\
        \rho^2\tau e & = \begin{bmatrix} -x \\ 1 \end{bmatrix} =  e
    \end{split}
\end{equation}
Evidently, every point on the edge is bijectively mapped to a point on the edge.

\newpage

\problem[\textit{Algebra} (Artin, 2e) Exercise 6.3.4 (extended)]
\subsection*{Part (a)}
Let $G = GL_n\left(\R\right)$ act on the set $V = \R^n$ by left multiplication. Since $g0 = 0$ for all $g \in G$ and the null space of an invertible matrix is only the zero vector, $O_0 = \left\{0\right\}$. Further, for some nonzero vectors $v$, there must exist some $g \in G$ such that $ge_1 = v$. Since $ge_1$ is the first column of $g$, we see that $g$ is a matrix where the first column is $v$ with the other columns chosen such that the matrix is invertible. Since $e_1$ can therefore be mapped to every nonzero vector in $\R^n$, $O_{e_1} = V - \left\{0\right\}$. The two orbits are then the set containing only the zero vector, and the set containing the rest of $V = \R^n$.
\subsection*{Part (b)}
The stabilizer of $e_1$ is the set of all $g \in G$ such that $ge_1 = e_1$. Since $ge_1$ is the first column of $g$, the stabilizer of $e_1$ is therefore the set of all invertible matrices where the first column is $e_1$.
\subsection*{Part (c)}
The action of $G$ on $V - \left\{0\right\}$ is then observed to have a single orbit. Consequently, for $v, v' \in V - \left\{0\right\}$, there exists $g \in G$ such that $gv = v'$, which means that the action is transitive. We see that for all $v \in V$, $Iv = v$. Suppose, for arbitrary $v$, there exists some $g' \in G$ such that $g'v = v = Iv$. Since a matrix is uniquely determined by the linear transformation it performs, $g' = I$. Consequently, $gv = v$ iff $g = I$, so the action is free.

\end{document}