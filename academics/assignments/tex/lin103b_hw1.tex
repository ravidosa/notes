\documentclass{article}
\usepackage{assignment_preamble}

\title{Homework 1}
\author{Ravi Kini}
\date{January 15, 2023}

\begin{document}

\maketitle

\problem[\textit{Syntactic Constructions in English} (Kim and Michaelis) Exercise 1.1]
\textbf{competence}: Native English speakers can determine whether a certain combination of English words is grammatical, even for sentences they have never encountered before.

\textbf{performance}: Although native English speakers have mental knowledge of English, they may still make error in their speech.

\textbf{descriptive}: While many consider the phrase \textit{ain't} to be grammatically incorrect, it can still be studied by looking at how it can be regarded as valid by mental grammar.

\textbf{prescriptive}: Ending a sentence with a proposition is considered ungrammatical, as it is not considered valid in "proper" English.

\textbf{language faculty}: In the Chomskyan view, part of the human brain is hardwired to process and interpret language.

\textbf{core}: English sentences can be broken down into noun and verb phrases, which themselves can be broken down into individual lexemes.

\textbf{peripheral}: Idioms do not have literal meanings, and fall outside of the universal grammar framework.

\textbf{inductive}: Analyzing a large corpus of English can allow one to draw conclusions about English grammar.

\textbf{deductive}: Applying a hypothetical rule of English grammar to sentences can allow one to determine whether or not the rule holds.

\textbf{nativist}: Chomsky theorizes that grammar is hardwired into children's brains, which results in them using language in ways that are correct syntactically, but not semantically.

\textbf{usage-based}: There is no clear line between syntax and semantics, as evidenced by the spectrum between core and peripheral patterns.
\clearpage

\problem[\textit{Syntactic Constructions in English} (Kim and Michaelis) Exercise 1.2 (partial)]
\textbf{knowledge}
\begin{exe}
    \ex[*] {I have a lot of knowledges.}
    \ex[*] {He got a knowledge.}
    \ex[*] {We do not have many knowledge.}
    \ex {We do not have much knowledge.}
    \ex[*] {She has few knowledge of this.}
    \ex {She has little knowledge of this.}
\end{exe}
The data indicates that \textit{knowledge} is used as a mass noun.

\textbf{discussion}
\begin{exe}
    \ex {I often attend discussions.}
    \ex {Have you missed a discussion?}
    \ex {Many discussions end with disagreement.}
    \ex {Much discussion has been had on the topic.}
    \ex {They have had a few discussions.}
    \ex {There was little discussion while eating.}
\end{exe}
The data indicates that \textit{discussion} is used as both a count noun and a mass noun.

\textbf{difficulty}
\begin{exe}
    \ex {He had difficulties speaking.}
    \ex {There's not a difficulty we can't face.}
    \ex {He had many difficulties with the idea.}
    \ex {I did it with much difficulty.}
    \ex {You must have had few difficulties in life.}
    \ex {She had little difficulty with it.}
\end{exe}
The data indicates that \textit{difficulty} is used as both a count noun and a mass noun.

\textbf{research}
\begin{exe}
    \ex[*] {She does a lot of researches.}
    \ex[*] {Have you ever done a research?}
    \ex[*] {He has done many research on the matter.}
    \ex {He has done much research on the matter.}
    \ex[*] {I've done few research.}
    \ex {I've done little research.}
\end{exe}
The data indicates that \textit{research} is used as a mass noun.
\clearpage

\problem[\textit{Syntactic Constructions in English} (Kim and Michaelis) Exercise 1.3 (partial)]
\subsection*{Part (a)}
The sentence is ungrammatical because of a lack of subject-verb agreement between the subject \textit{Spring and fall}, which is plural, and the verb \textit{is},  which is singular.
\subsection*{Part (c)}
The sentence is grammatical.
\subsection*{Part (e)}
The sentence is ungrammatical because either \textit{he} should be replaced by \textit{it}, or \textit{her} should be removed.
\subsection*{Part (g)}
The sentence is ungrammatical because \textit{and} should be replaced by \textit{with}.
\clearpage

\problem[\textit{Syntactic Constructions in English} (Kim and Michaelis) Exercise 1.4]
The reflexive pronoun should agree with the pronoun that it refers to. In (i), only (a) is grammatical because it is the only sentence where the reflexive pronoun in the object, \textit{himself}, agrees with the personal pronoun that it refers to in the subject, \textit{He}. In (ii), only (a) is ungrammatical because the personal pronoun in the subject,  \textit{He}, and the personal pronoun in the object, \textit{he}, refer to the same person, so the object should have the reflexive form, as in (ia). In (iii) and (iv), the unrealized subject is \textit{you}, either in the singular or plural. Like (i.a), (iii.a) and (iii.b) are grammatical because the pronoun of the unrealied subject, \textit{you}, agrees with the reflexive pronouns in the object, \textit{yourself} and \textit{yourselves}. Like (ii.a), (iv.a) is ungrammatical because the pronoun of the unrealized subject, \textit{you}, and the personal pronoun in the object,  \textit{you}.
\clearpage

\problem[\textit{Syntactic Constructions in English} (Kim and Michaelis) Exercise 1.5 (partial)]
\subsection*{Part (a)}
The speaker is not literally asking what activity the elephants are engaged in, but instead indicating that this situation is unexpected (pragmatic).
\subsection*{Part (b)}
\textit{Just because} is not expressing that there is a causal relationship, but that it is insufficient for something to be the case (semantic).
\subsection*{Part (d)}
\textit{shoulder} does not involve any actions, but it is inflected as a verb to mean "pushing using on'e shoulders" (syntactic).

\end{document}