\documentclass{article}
\usepackage{assignment_preamble}

\title{Homework 3}
\author{Ravi Kini}
\date{January 28, 2024}

\begin{document}

\maketitle

\problem[\textit{Syntactic Constructions in English} (Kim and Michaelis, 1e) Exercise 3.1 (partial)]
\subproblem{(a)}
\begin{exe}
    \ex {I rode the bike.}
\end{exe}
\subproblem{(c)}
\begin{exe}
    \ex {They stopped by last Sunday.}
\end{exe}
\subproblem{(e)}
\begin{exe}
    \ex {She robbed them of their valuables.}
\end{exe}
\subproblem{(g)}
\begin{exe}
    \ex {He found her fascinating.}
\end{exe}
\clearpage

\problem[\textit{Syntactic Constructions in English} (Kim and Michaelis, 1e) Exercise 3.2 (partial)]
\subproblem{(b)}
\textit{this year} has the grammatical function of direct object.
\subproblem{(d)}
\textit{in little jars} has the grammatical function of oblique complement.
\subproblem{(f)}
\textit{That he does it with such a deft sense of equilibrium} has the grammatical function of subject.
\subproblem{(h)}
\textit{in little spring outfits} has the grammatical function of modifier.
\clearpage

\problem[\textit{Syntactic Constructions in English} (Kim and Michaelis, 1e) Exercise 3.5 (partial)]
\subproblem{(a)}
\begin{exe}
    \ex {He proved his innocence.}
    \ex {His innocence was proven by him.}
\end{exe}
\textit{his innocence} has the grammatical function of direct object.
\subproblem{(b)}
\begin{exe}
    \ex {He proved an adequate student.}
    \ex[*] {An adequate student was proven by him.}
\end{exe}
\textit{an adequate student} has the grammatical function of predicative complement.
\subproblem{(e)}
\begin{exe}
    \ex {Tori considered him a decent man.}
    \ex[*] {Tori considered him.}
    \ex {Tori considered him to be a decent man.}
\end{exe}
\textit{a decent man} has the grammatical function of predicative complement.
\subproblem{(f)}
\begin{exe}
    \ex {Tori cooked him a decent meal.}
    \ex[*] {Tori cooked him.}
    \ex {Tori cooked a decent meal for him.}
\end{exe}
\textit{a decent meal} has the grammatical function of direct object.
\clearpage

\problem[\textit{Syntactic Constructions in English} (Kim and Michaelis, 1e) Exercise 3.7 (partial)]
\subproblem{(a)}
\textit{John} has the semantic role of experiencer, and \textit{the freshly baked bread} has the semantic role of theme.
\subproblem{(c)}
\textit{Thomas Harty} has the semantic role of patient, and \textit{the knife} has the semantic role of instrument.
\subproblem{(e)}
\textit{Wright} has the semantic role of experiencer, and \textit{the full rumor} has the semantic role of theme.
\subproblem{(g)}
\textit{James} has the semantic role of agent, \textit{me} has the semantic role of benefactive, and \textit{a fruitcake} has the semantic role of theme.
\subproblem{(i)}
\textit{The teakettle} has the semantic role of theme, and \textit{the stove} has the semantic role of location.
\clearpage

\end{document}