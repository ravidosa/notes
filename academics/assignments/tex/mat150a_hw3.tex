\documentclass{article}
\usepackage{assignment_preamble}

\title{HW 3}
\author{Ravi Kini}
\date{October 24, 2023}

\begin{document}

\maketitle

\problem
Let $\varphi: G \to H$ be an isomorphism. For some $g \in G$, we first show that $\phi\left(g^n\right) = \phi\left(g\right)^n$. Since $\phi\left(g^1\right) = \phi\left(g\right) = \phi\left(g\right)^1$, the assertion clearly holds for $n = 1$. Assume this assertion holds for some $n$. Then:
\begin{equation}
    \begin{split}
        \phi\left(g^{n+1}\right) = \phi\left(g^ng\right) & = \phi\left(g^n\right)\phi\left(g\right) \\
        & = \phi\left(g\right)^n\phi\left(g\right) = \phi\left(g\right)^{n+1}
    \end{split}
\end{equation}
By the Principle of Mathematical Induction, $\phi\left(g^n\right) = \phi\left(g\right)^n$ for all $n \in \N$.
Suppose $\ord\left(g\right) = \infty$ and $\ord\left(\varphi\left(g\right)\right) = n < \infty$. Then:
\begin{equation}
    \begin{split}
        \varphi\left(g\right)^n & = 1_H \\
        \varphi\left(g^n\right) & = \varphi\left(1_G\right) \\
    \end{split}
\end{equation}
Since  $\varphi$ is an isomorphism and therefore biijective, this means $g^n = 1_G$, which is a contradiction. Now suppose $\ord\left(g\right) = n < \infty$ and $\ord\left(\varphi\left(g\right)\right) = \infty$. Then:
\begin{equation}
    \begin{split}
        \varphi\left(g^n\right) & = \varphi\left(1_G\right) = 1_H \\
        \varphi\left(g\right)^n & = \\
    \end{split}
\end{equation}
This means $\ord\left(\varphi\left(g\right)\right) = n < \infty$, which is a contradiction. $\ord\left(g\right)$ and $\ord\left(\varphi\left(g\right)\right)$ are then either both finite or both infinite. If both are infinite, then $\ord\left(g\right) = \ord\left(\varphi\left(g\right)\right)$. Now suppose both are finite, such that $\ord\left(g\right) = n$ and $\ord\left(\varphi\left(g\right)\right) = m$. Then:
\begin{equation}
    \begin{split}
        \varphi\left(g\right)^n = \varphi\left(g^n\right) & = \varphi\left(1_G\right) = 1_H
    \end{split}
\end{equation}
Since $m$ is the smallest positive integer such that $\varphi\left(g\right)^m = 1_H$, $m \leq n$. Furthermore:
\begin{equation}
    \begin{split}
        \varphi\left(g^m\right) = \varphi\left(g\right)^m & = 1_H = \varphi\left(1_G\right)
    \end{split}
\end{equation}
Since  $\varphi$ is an isomorphism and therefore biijective, this means $g^m = 1_G$. Since $n$ is the smallest positive integer such that $g^n = 1_G$, $n \leq m$. Consequently, $m = n$ and $\ord\left(g\right) = \ord\left(\varphi\left(g\right)\right)$. In all cases, $\ord\left(g\right) = \ord\left(\varphi\left(g\right)\right)$.

\newpage

\problem
\subsection*{Part (a)}
Let $\left(A,\star\right)$ and $\left(B, \diamond\right)$ be groups, and let $A\times B$ be their direct product. Let $a_1, a_2, a_3 \in A$ and $b_1, b_2, b_3 \in B$. Then:
\begin{equation}
    \begin{split}
        \left(\left(a_1,b_1\right)\left(a_2,b_2\right)\right)\left(a_3,b_3\right) & = \left(a_1 \star a_2, b_1 \diamond b_2\right)\left(a_3,b_3\right) \\
        & = \left(a_1 \star a_2  \star a_3, b_1 \diamond b_2 \diamond b_3\right) \\
        \left(a_1,b_1\right)\left(\left(a_2,b_2\right)\left(a_3,b_3\right)\right) & = \left(a_1, b_1\right)\left(a_2 \star a_3, b_2 \diamond b_3\right) \\
        & = \left(a_1 \star a_2  \star a_3, b_1 \diamond b_2 \diamond b_3\right) \\
    \end{split}
\end{equation}
Since $\left(\left(a_1,b_1\right)\left(a_2,b_2\right)\right)\left(a_3,b_3\right) = \left(a_1,b_1\right)\left(\left(a_2,b_2\right)\left(a_3,b_3\right)\right)$ for all $a_1, a_2, a_3 \in A$ and $b_1, b_2, b_3 \in B$, multiplication is associative.
\subsection*{Part (b)}
Let $1_A$ be the identity element in $A$ and $1_B$ be the identity element in $B$. Then:
\begin{equation}
    \begin{split}
        \left(a,b\right)\left(1_A,1_B\right) & = \left(a \star 1_A, b \diamond 1_B\right) \\
        & = \left(a, b\right)
    \end{split}
\end{equation}
Since $\left(a,b\right)\left(1_A,1_B\right) = \left(a, b\right)$, the identity element in $A \times B$ is $\left(1_A, 1_B\right)$. 
\subsection*{Part (c)}
Let $a^{-1}$ be the inverse of $a$ in $A$ and $b^{-1}$ be the inverse of $b$ in $B$. Then:
\begin{equation}
    \begin{split}
        \left(a,b\right)\left(a^{-1},b^{-1}\right) & = \left(a \star a^{-1}, b \diamond b^{-1}\right) \\
        & = \left(1_A, 1_B\right)
    \end{split}
\end{equation}
Since $\left(a,b\right)\left(a^{-1},b^{-1}\right) = \left(1_A, 1_B\right)$, the inverse of $\left(a, b\right)$ in $A \times B$ is $\left(a^{-1}, b^{-1}\right)$. 

\newpage

\problem
\subsection*{Part (a)}
Let $g$ be the generator of the cyclic group $C_p$ and $\varphi$ be an automorphism of $C_p$. Let $\varphi\left(g\right) = g^i$. Then, as automorphisms are a type of isomorphism and using the result found as part of Exercise 1:
\begin{equation}
    \begin{split}
        \varphi\left(g^j\right) = \varphi\left(g\right)^j & = \left(g^i\right)^j
    \end{split}
\end{equation}
Since $\varphi$ is an automorphism, $C_p = \left\{\left(g^i\right)^j : j \in \Z \right\}$, which means that $g^i$ is a generator of $C_p$. Evidently $C_p$ has as many automorphisms as there are ways to map $g$ to a generator of $C_p$, which is equal to the number of generators of $C_p$.

We assert that $g^i$ is a generator of $C_p$ iff $i$ and $p$ are relatively prime. Let $g^i$ be a generator of $C_p$ and suppose $i$ and $p$ are not relatively prime. There then exists some integer $n > 1$ such that $an = i$ and $bn = p$ for $a, b \in \Z$. Then:
\begin{equation}
    \begin{split}
        \left(g^{i}\right)^b = g^{ib} = g^{anb} & = g^{ap} = \left(g^p\right)^a \\
        & = 1_G^a = 1_G
    \end{split}
\end{equation}
The order of the cyclic subgroup generated by $g^i$ is then at most $b$. However, since $b < bn = p$, the cyclic subgroup generated by $g^i$ cannot be $C_p$, which has order $p$. Consequently, $g^i$ cannot be a generator of $C_p$. This is a contradiction; therefore, $i$ and $p$ are relatively prime. Now let $g^i \in C_p$ such that $i$ and $p$ are relatively prime. The order of the cyclic subgroup generated by $g^i$ is obviously at most $p$, as otherwise $G$ would no longer be a group. Let $\langle g^i \rangle$ have order $j < p$. Then $\left(g^i\right)^j = g^{ij}$, which means that $ij = kp$ for some $k \in \Z$. Since $i$ and $p$ are relatively prime, $p$ must divide $j$. This is a contradiction, as $j < p$; therefore $\ord\left(\langle g^i \rangle\right) \geq p$. Consequently, $\ord\left(\langle g^i \rangle\right) = p$, which means that $g^i$ is a generator of $C_p$. Evidently, $g^i$ is a generator of $C_p$ iff $i$ and $p$ are relatively prime.

For prime $p$, there are $p - 1$ relatively prime integers less than $p$. $C_p$ therefore has $p - 1$ automorphisms.
\subsection*{Part (b)}
From the results of Part (a), since there are 8 integers less than 24 that are relatively prime to 24, there are 8 automorphisms of $C_{24}$.

\newpage

\problem[\textit{Algebra} (Artin, 2e) Exercise 2.5.2]
\subsection*{Part (a)}
Let $K, H \leq G$ for some group $G$. Since the identity of a group is the identity of the subgroup, $1_G$ is the identity element in both $K$ and $H$, and is therefore an element of $K \cap H$. Let $a, b \in K \cap H$. Since $a, b \in K \cap H$, $a, b \in K$ and therefore, since $K$ is a subgroup fo $G$, $ab \in K$. Similarly, $ab \in H$, which means $ab \in K \cap H$. Let $c \in K \cap H$. Since $c \in K \cap H$, $c \in K$ and therefore, since $K$ is a subgroup of $G$, $c^{-1} \in K$. Similarly, $c^{-1} \in H$, which means $c^{-1} \in K \cap H$. Since the identity, closure, and inverse properties hold, $K \cap H$ is a subgroup of $G$.
\subsection*{Part (b)}
From Part (a), since $K \cap H \subseteq H$ and the identity, closure, and inverse properties hold, $K \cap H$ is a subgroup of $H$. Let $k' \in K \cap H$ and $h \in H$. Since $k' \in K$ and $h \in G$, $hk'h^{-1} \in K$ since $K \normal G$. Furthermore, since $k' \in H$, $hk'h^{-1} \in H$ due to the closure of subgroups. Therefore $hk'h^{-1} \in K \cap H$ and $K \cap H \normal H$.

\newpage

\problem[\textit{Algebra} (Artin, 2e) Exercise 2.6.4]
Let $a, b \in G$ for some group $G$. Since $G$ is a group, $a^{-1}, b^{-1} \in G$. Then:
\begin{equation}
    \begin{split}
        a^{-1}\left(ab\right)a & = \left(a^{-1}a\right)\left(ba\right) \\
        & = ba \\
        b^{-1}\left(ba\right)b & = \left(b^{-1}b\right)\left(ab\right) \\
        & = ab
    \end{split}
\end{equation}
Since $a^{-1}, b^{-1} \in G$, $ba$ is the conjugate of $ab$ by $a^{-1}$ and $ab$ is the conjugate of $ba$ by $b^{-1}$, which means $ab$ and $ba$ are conjugate elements.

\newpage

\problem[\textit{Algebra} (Artin, 2e) Exercise 2.8.10 (partial)]
Let $H \leq G$ for some group $G$ such that $[G : H] = 2$. $H$ then has two left cosets and two right cosets in $G$. Let $g \in G$. Suppose $g \in H$. Then $gH = H = Hg$. How suppose $g \in G \setminus H$. Then $gH = G \setminus H$ since the two cosets of $H$ partition $G$. Similarly, $Hg = G \setminus H$, which means $gH = Hg$ for all $g \in G$. Consequently, $gHg^{-1} = H$, which means $H \normal G$.


\end{document}