\documentclass{article}
\usepackage{assignment_preamble}

\title{Homework 2}
\author{Ravi Kini}
\date{February 21, 2024}

\begin{document}

\maketitle

\problem[\textit{Introduction to Elementary Particles} (Griffiths, 2e) Exercise 3.7]
\begin{equation}
    \begin{split}
        x^{\mu} & = M_{\nu}^{\mu}{x^{\nu}}' \\
        M & = \begin{pmatrix}
            \gamma & \gamma\beta & 0 & 0 \\
            \gamma\beta & \gamma & 0 & 0 \\
            0 & 0 & 1 & 0 \\
            0 & 0 & 0 & 1 \\
        \end{pmatrix} \\
        \Lambda M & = \begin{pmatrix}
            \gamma & -\gamma\beta & 0 & 0 \\
            -\gamma\beta & \gamma & 0 & 0 \\
            0 & 0 & 1 & 0 \\
            0 & 0 & 0 & 1 \\
        \end{pmatrix}\begin{pmatrix}
            \gamma & \gamma\beta & 0 & 0 \\
            \gamma\beta & \gamma & 0 & 0 \\
            0 & 0 & 1 & 0 \\
            0 & 0 & 0 & 1 \\
        \end{pmatrix} \\
        & = \begin{pmatrix}
            \gamma^2(1 - \beta^2) & 0 & 0 & 0 \\
            0 & \gamma^2(1 - \beta^2) & 0 & 0 \\
            0 & 0 & 1 & 0 \\
            0 & 0 & 0 & 1 \\
        \end{pmatrix} \\
        & = \begin{pmatrix}
            1 & 0 & 0 & 0 \\
            0 & 1 & 0 & 0 \\
            0 & 0 & 1 & 0 \\
            0 & 0 & 0 & 1 \\
        \end{pmatrix} = I
    \end{split}
\end{equation}
$M$ is then the matrix inverse of the Lorentz transformation matrix $\Lambda$.
\clearpage

\problem[\textit{Introduction to Elementary Particles} (Griffiths, 2e) Exercise 3.12]
Applying the Lorentz transformation:
\begin{equation}
    \begin{split}
        p_A^{\mu} + p_B^{\mu} & = p_C^{\mu} + p_D^{\mu} \\
        \Lambda_{\nu}^{\mu}\left(p_A^{\mu} + p_B^{\mu}\right) & = \Lambda_{\nu}^{\mu}\left(p_C^{\mu} + p_D^{\mu}\right) \\
        {p_A^{\nu}}' + {p_B^{\nu}}' & = {p_C^{\nu}}' + {p_D^{\mu}}' \\
    \end{split}
\end{equation}
Energy and momentum are conserved in $S'$.
\clearpage

\problem[\textit{Introduction to Elementary Particles} (Griffiths, 2e) Exercise 3.15]
By conservation of the energy-momentum 4-vector:
\begin{equation}
    \begin{split}
        p_{\pi} & = p_{\mu} + p_{\nu} \\
        p_{\mu} & = p_{\pi} - p_{\nu} \\
        p_{\mu}^2 & = p_{\pi}^2 + p_{\nu}^2 - 2p_{\pi} \cdot p_{\nu} \\
        m_{\mu}^2c^2 & = m_{\pi}^2c^2 - 2\left(\frac{E_{\pi}}{c}\frac{E_{\nu}}{c} - \mathbf{p}_{\pi} \cdot \mathbf{p}_{\nu}\right) \\
        & = m_{\pi}^2c^2 - 2\frac{E_{\pi}E_{\nu}}{c^2} \\
        & = m_{\pi}^2c^2 - 2\gamma m_{\pi}\left|\mathbf{p}_{\nu}\right|c \\
        \left|\mathbf{p}_{\nu}\right| & = \frac{(m_{\pi}^2 - m_{\mu}^2)c}{2\gamma m_{\pi}} \\
    \end{split}
\end{equation}
Calculating the scattering angle:
\begin{equation}
    \begin{split}
        \tan\theta & = \frac{\left|\mathbf{p}_{\nu}\right|}{\left|\mathbf{p}_{\pi}\right|} = \frac{\frac{\left(m_{\pi}^2 - m_{\mu}^2\right)c}{2\gamma m_{\pi}}}{\gamma m_{\pi}\beta c} \\
        & = \frac{m_{\pi}^2 - m_{\mu}^2}{2\beta \left(\gamma m_{\pi}\right)^2} \\
        & = \frac{1 - \left(\frac{m_{\mu}}{m_{\pi}}\right)^2}{2\beta \gamma^2} \\
        \theta & = \tan^{-1} \frac{1 - \left(\frac{m_{\mu}}{m_{\pi}}\right)^2}{2\beta \gamma^2}
    \end{split}
\end{equation}
\clearpage

\problem[\textit{Introduction to Elementary Particles} (Griffiths, 2e) Exercise 3.16]
Before the collision:
\begin{equation}
    \begin{split}
        p^{\mu} & = \left(\frac{E_A}{c} + m_Bc, \mathbf{p}_A\right) \\
        p^2 & = \left(\frac{E_A}{c} + m_Bc\right)^2 -  \mathbf{p}_A^2 \\
        & = \frac{E_A^2}{c^2} + 2E_am_B + m_B^2c^2 - \frac{1}{c^2}\left(E_A^2 - m_A^2c^4\right) \\
        & = 2E_am_B + \left(m_B^2 + m_A^2\right)c^2 \\
    \end{split}
\end{equation}
After the collision:
\begin{equation}
    \begin{split}
        {p^{\mu}}' & = \left(\sum_{i=1}^n m_nc, \mathbf{0}\right) = \left(Mc, \mathbf{0}\right) \\
    \end{split}
\end{equation}
Using the invariance of the dot product:
\begin{equation}
    \begin{split}
        p^2 = {p^2}' & = M^2c^2 \\
        E_a & = \frac{M^2 - m_B^2 - m_A^2}{2m_B}c^2
    \end{split}
\end{equation}
\clearpage

\problem[\textit{Introduction to Elementary Particles} (Griffiths, 2e) Exercise 3.17]
\subproblem{(a)}
Applying the previous results to $p + p \to p + p + \pi^0$:
\begin{equation}
    \begin{split}
        E_p & = \frac{\left(2m_p + m_{\pi^0}\right)^2 - m_p^2 - m_p^2}{2m_p}c^2 \\
        & \approx 1217.94~\unit{MeV}
    \end{split}
\end{equation}
\subproblem{(b)}
Applying the previous results to $p + p \to p + p + \pi^+ + \pi^-$:
\begin{equation}
    \begin{split}
        E_p & = \frac{\left(2m_p + m_{\pi^+} + m_{\pi^-}\right)^2 - m_p^2 - m_p^2}{2m_p}c^2 \\
        & \approx 1538.07~\unit{MeV}
    \end{split}
\end{equation}
\subproblem{(c)}
Applying the previous results to $\pi^- + p \to p + \bar{p} + n$:
\begin{equation}
    \begin{split}
        E_{\pi^-} & = \frac{\left(m_p + m_{\bar{p}} + m_{n}\right)^2 - m_p^2 - m_{\pi^{-}}^2}{2m_p}c^2 \\
        & \approx 3746.60~\unit{MeV}
    \end{split}
\end{equation}
\subproblem{(d)}
Applying the previous results to $\pi^- + p \to K^0 + \Sigma^0$:
\begin{equation}
    \begin{split}
        E_{\pi^-} & = \frac{\left(m_{K^0} + m_{\Sigma^0}\right)^2 - m_p^2 - m_{\pi^{-}}^2}{2m_p}c^2 \\
        & \approx 1042.94~\unit{MeV}
    \end{split}
\end{equation}
\subproblem{(e)}
Applying the previous results to $p + p \to p + \Sigma^+ + K^0$:
\begin{equation}
    \begin{split}
        E_p & = \frac{\left(m_p + m_{\Sigma^+} + m_{K^0}\right)^2 - m_p^2 - m_p^2}{2m_p}c^2 \\
        & \approx 2734.61~\unit{MeV}
    \end{split}
\end{equation}
\clearpage

\problem[\textit{Introduction to Elementary Particles} (Griffiths, 2e) Exercise 3.18]
For the second collision:
\begin{equation}
    \begin{split}
        E_{K^-} & = \frac{\left(m_{\Omega^-} + m_{K^0} + m_{K^+}\right)^2 - m_p^2 - m_{K^-}^2}{2m_p}c^2 \\
        & \approx 3182.32~\unit{MeV} \\
    \end{split}
\end{equation}
For the first collision, treating the non-$K^-$ products as a particle $P$:
\begin{equation}
    \begin{split}
        p_{p_1} + p_{p_2} & = p_P + p_{K^-} \\
        \left(p_{p_1} - p_{K^-}\right)^2 & = \left(p_P - p_{p_2}\right)^2 \\
        & m_{p_1}^2c^2 + m_{K^-}^2c^2 -2\left(\frac{E_p}{c}\frac{E_{K^-}}{c} - \mathbf{p}_{p_1} \cdot \mathbf{p}_{K^-}\right) \\
        & = m_{P}^2c^2 + m_{p_2}^2c^2 -2\left(\frac{E_P}{c}m_pc\right) \\
        & m_{K^-}^2c^2 -2\left(\frac{E_pE_{K^-} - \sqrt{\left(E_p^2 - m_p^2c^4\right)\left(E_{K^-}^2 - m_{K^-}^2c^4\right)}}{c^2}\right) \\
        & = m_{P}^2c^2 -2m_p(E_p + m_pc^2 - E_{K^-}) \\
        & E_p\left(E_{K^-} - m_pc^2\right) + \frac{m_{P}^2 - m_{K^-}^2 - 2m_p^2 + 2m_p\frac{E_{K^-}}{c^2}}{2}c^4 \\
        & = \sqrt{\left(E_p^2 - m_p^2c^4\right)\left(E_{K^-}^2 - (m_{K^-}c)^2\right)} \\
    \end{split}
\end{equation}
Let $a = E_{K^-} - m_pc^2, b = \frac{m_{P}^2 - m_{K^-}^2 - 2m_p^2 + 2m_p\frac{E_{K^-}}{c^2}}{2}c^4, d = E_{K^-}^2 - \left(m_{K^-}c\right)^2, e = m_p^2c^4$.
\begin{equation}
    \begin{split}
        a & = E_{K^-} - m_pc^2 \\
        & \approx 2244.05~\unit{MeV} \\
        b & = \frac{m_{P}^2 - m_{K^-}^2 - 2m_p^2 + 2m_p\frac{E_{K^-}}{c^2}}{2}c^4 \\
        & \approx 4792636.24~\unit{MeV\squared} \\
        d & = E_{K^-}^2 - (m_{K^-}c)^2 \\
        & \approx 9883440.64~\unit{MeV\squared} \\
        e & = (m_pc)^2 \\
        & \approx 880350.59~\unit{MeV\squared} \\
        aE_p + b & = \sqrt{(E_p^2 - (m_pc)^2)d} \\
        a^2E_p^2 + 2abE_p + b^2 & = (E_p^2 - e)d \\
        0 & = (a^2 - d)E_p^2 + (2ab)E_p + (b^2 + de) \\
        E_p & = \frac{-2ab \pm \sqrt{(2ab)^2 - 4(a^2 - d)(b^2 + de)}}{2(a^2 - d)} \\
        & = -1165.97, 5603.11~\unit{MeV}
    \end{split}
\end{equation}
The incident proton has minimum kinetic energy $E_p - m_pc^2 = 5603.11 - 938.27 = 4664.84~\unit{MeV}$.
\clearpage

\problem[\textit{Introduction to Elementary Particles} (Griffiths, 2e) Exercise 3.21]
Applying conservation of the energy-momentum 4-vector:
\begin{equation}
    \begin{split}
        \mathbf{p}_{\pi^-} = 0 & = \mathbf{p}_{\mu^-} + \mathbf{p}_{\nu} \\
        \mathbf{p}_{\mu^-} = \mathbf{p} & = -\mathbf{p}_{\nu} \\
        E_{\pi^-} & = E_{\mu^-} + E_{\bar{\nu}_\mu} \\
        m_{\pi^-}c^2 & = c\sqrt{(m_{\mu^-}c)^2 + \mathbf{p}^2} + |\mathbf{p}|c \\
        |\mathbf{p}_\mu| = |\mathbf{p}| & = \frac{m_{\pi^-}^2 - m_{\mu^-}^2}{2m_{\pi^-}}c \\
        E_\mu & = \sqrt{m_{\mu}^2c^4 + \mathbf{p}^2c^2} = \frac{m_{\pi^-}^2 + m_{\mu^-}^2}{2m_{\pi^-}}c^2 \\
        v_\mu & = \frac{\mathbf{p}_{\mu}c^2}{E} = \frac{m_{\pi^-}^2 - m_{\mu^-}^2}{m_{\pi^-}^2 + m_{\mu^-}^2}c \\
        \gamma & = \frac{1}{\sqrt{1 - (\frac{v}{c})^2}} = \frac{1}{\sqrt{1 - (\frac{m_{\pi^-}^2 - m_{\mu^-}^2}{m_{\pi^-}^2 + m_{\mu^-}^2})^2}} = \frac{m_{\pi^-}^2 + m_{\mu^-}^2}{2m_{\pi^-}m_{\mu^-}} \\
        d & = v\gamma\tau = \frac{m_{\pi^-}^2 - m_{\mu^-}^2}{m_{\pi^-}^2 + m_{\mu^-}^2}c\frac{m_{\pi^-}^2 + m_{\mu^-}^2}{2m_{\pi^-}m_{\mu^-}}\tau \\
        & = \frac{m_{\pi^-}^2 - m_{\mu^-}^2}{2m_{\pi^-}m_{\mu^-}}c\tau \approx 185.87~\unit{\meter}
    \end{split}
\end{equation}
\clearpage

\problem[\textit{Introduction to Elementary Particles} (Griffiths, 2e) Exercise 3.27]
Applying conservation of the energy-momentum 4-vector:
\begin{equation}
    \begin{split}
        p\sin\phi & = p_\gamma'\sin\theta \\
        \sin\phi & = \frac{E'}{pc}\sin\theta \\
        p_\gamma & = p\cos\phi + p_\gamma'\cos\theta \\
        \frac{E}{c} & = p\sqrt{1 - (\frac{E'}{pc}\sin\theta)^2} + \frac{E'}{c}\cos\theta \\
        & = \sqrt{p^2 - (\frac{E'}{c}\sin\theta)^2} + \frac{E'}{c}\cos\theta \\
        p & = (\frac{E}{c} - \frac{E'}{c}\cos\theta)^2 + (\frac{E'}{c}\sin\theta)^2 \\
        p^2c^2 & = E^2 - 2EE'\cos\theta + E'^2 \\
        E + mc^2 & = \sqrt{m^2c^4 + p^2c^2} + E' \\
        E + mc^2 - E' & = \sqrt{m^2c^4 + E^2 - 2EE'\cos\theta + E'^2} \\
        -EE' + (E - E')mc^2 & = -EE'\cos\theta \\
        E' & = \frac{Emc^2}{E(1 - \cos\theta) + mc^2} \\
        \frac{hc}{\lambda'} & = \frac{\frac{hc}{\lambda}mc^2}{\frac{hc}{\lambda}(1 - \cos\theta) + mc^2} \\
        \lambda' & = \frac{hc(1 - \cos\theta) + mc^2\lambda}{mc^2} \\
        & = \lambda + \frac{h}{mc}(1 - \cos\theta)
    \end{split}
\end{equation}

\end{document}