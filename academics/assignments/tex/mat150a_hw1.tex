\documentclass{article}
\usepackage{assignment_preamble}

\title{Homework 1}
\author{Ravi Kini}
\date{October 10, 2023}

\begin{document}

\maketitle

\problem
\subsection*{Part (a)}
The matrices:
\begin{center}
    $p = 
    \begin{pmatrix} 
        1 & 0 & 0 \\
        0 & \cos\theta & \sin\theta  \\
        0 & -\sin\theta & \cos\theta
    \end{pmatrix}$
    \hspace{.25in}
    $y = 
    \begin{pmatrix} 
        \cos\theta & 0 & -\sin\theta \\
        0 & 1 &  0 \\
        \sin\theta & 0 & \cos\theta
    \end{pmatrix}$
    \hspace{.25in}
    $r = 
    \begin{pmatrix} 
        \cos\theta & -\sin\theta & 0 \\
        \sin\theta & \cos\theta  & 0 \\
        0 & 0 & 1
    \end{pmatrix}
    $
\end{center}
are the generators of the group $G$ of rotations of $\theta = \frac{\pi}{4}$ about the conventional $x$-, $y$-, and $z-$ axes, respectively. Since $\theta = \frac{\pi}{4}$, $\cos \theta = \sin \theta = \frac{1}{\sqrt{2}} := t$, we can rewrite our matrices as:
\begin{center}
    $p = 
    \begin{pmatrix} 
        1 & 0 & 0 \\
        0 & t & t  \\
        0 & -t & t
    \end{pmatrix}$
    \hspace{.25in}
    $y = 
    \begin{pmatrix} 
        t & 0 & -t \\
        0 & 1 &  0 \\
        t & 0 & t
    \end{pmatrix}$
    \hspace{.25in}
    $r = 
    \begin{pmatrix} 
        t & -t & 0 \\
        t  & t  & 0 \\
        0 & 0 & 1
    \end{pmatrix}
    $
\end{center}
We seek to find matrices representing $p^{-1}$ and $y^{-1}$; spatial reasoning indicates that the inverse of a rotation by an angle $\theta$ would be a rotation by an angle $-\theta$. The matrices representing $p^{-1}$ and $y^{-1}$ are then:
\begin{equation}
    \begin{split}
        p^{-1} & = \begin{pmatrix} 
            1 & 0 & 0 \\
            0 & \cos -\theta & \sin -\theta  \\
            0 & -\sin -\theta & \cos -\theta
        \end{pmatrix}  = \begin{pmatrix} 
            1 & 0 & 0 \\
            0 & \cos \theta & -\sin \theta  \\
            0 & \sin \theta & \cos \theta
        \end{pmatrix}  = \begin{pmatrix} 
            1 & 0 & 0 \\
            0 & t & -t  \\
            0 & t & t
        \end{pmatrix} \\
        y^{-1} & = 
        \begin{pmatrix} 
            \cos -\theta  & 0 & -\sin -\theta \\
            0 & 1 &  0 \\
            \sin -\theta & 0 & \cos -\theta
        \end{pmatrix} = 
        \begin{pmatrix} 
            \cos \theta  & 0 & \sin \theta \\
            0 & 1 &  0 \\
            -\sin \theta & 0 & \cos \theta
        \end{pmatrix} = 
        \begin{pmatrix} 
            t  & 0 & t \\
            0 & 1 &  0 \\
            -t & 0 & t
        \end{pmatrix}
    \end{split}
\end{equation}
To verify that these are the inverses of $p$ and $y$, respectively, we multiply $p$ and $p^{-1}$, and $y$ and $y^{-1}$.
\begin{equation}
    \begin{split}
        pp^{-1} & = \begin{pmatrix} 
            1 & 0 & 0 \\
            0 & t & t  \\
            0 & -t & t
        \end{pmatrix}\begin{pmatrix} 
            1 & 0 & 0 \\
            0 & t & -t  \\
            0 & t & t
        \end{pmatrix} = \begin{pmatrix} 
            1 & 0 & 0 \\
            0 & 2t^2 & 0  \\
            0 & 0 & 2t^2
        \end{pmatrix} = \begin{pmatrix} 
            1 & 0 & 0 \\
            0 & 1 & 0  \\
            0 & 0 & 1
        \end{pmatrix} \\
        yy^{-1} & = \begin{pmatrix} 
            t & 0 & -t \\
            0 & 1 &  0 \\
            t & 0 & t
        \end{pmatrix}\begin{pmatrix} 
            t & 0 & t \\
            0 & 1 &  0 \\
            -t & 0 & t
        \end{pmatrix} = \begin{pmatrix} 
            2t^2 & 0 & 0 \\
            0 & 1 & 0  \\
            0 & 0 & 2t^2
        \end{pmatrix} = \begin{pmatrix} 
            1 & 0 & 0 \\
            0 & 1 & 0  \\
            0 & 0 & 1
        \end{pmatrix} \\
    \end{split}
\end{equation}
As they multiply to the $3 \times 3$ identity matrix, they must be inverses.
\subsection*{Part (b)}
We now compute $p^2y^{-1}p^{-2}$ and $py^{-1}p^{-1}$:
\begin{equation}
    \begin{split}
        p^2y^{-1}p^{-2} & = 
        \begin{pmatrix} 
            t & -t & 0 \\
            t & t & 0  \\
            0 & 0 & 1
        \end{pmatrix} = r \\
        py^{-1}p & =  
        \begin{pmatrix} 
            t & -t^2 & -t^2 \\
            -t^2 & t^2 - t^3 & -t^2 - t^3 \\
            t^2 & t^2 + t^3 & -t^2 + t^3
        \end{pmatrix} \neq r \\
    \end{split}
\end{equation}
$p^2y^{-1}p^{-2} = r$, but $py^{-1}p^{-1} \neq r$.

\clearpage

\problem
\subsection*{Part (a)}
The set of transpositions $\left\{\tau_1,\tau_2,\tau_3\right\}$ generate $S_4$, where $\tau_i = \left(i~ i+1\right)$. The permutations:
\begin{center}
    $p = \left(1~2~3~4\right)
    \qquad
    q = \left(1~3~2~4\right)
    \qquad
    r = \left(1~4~2\right)
    $
\end{center}
can therefore be written as products of (adjacent) transpositions $\tau_i$ in the following way:
\begin{equation}
    \begin{split}
        \tau_1 \circ \tau_2 \circ \tau_3 & = \tau_1 \circ \left(2~3\right) \circ \left(3~4\right) \\
        & = \left(1~2\right) \circ \left(2~3~4\right) \\
        & = \left(1~2~3~4\right) = p \\
        \tau_2 \circ \tau_1 \circ \tau_3 \circ \tau_2 \circ \tau_3 & = \tau_2 \circ \tau_1 \circ \tau_3 \circ \left(2~3\right) \circ \left(3~4\right) \\
        & = \tau_2 \circ \tau_1 \circ \left(3~4\right) \circ \left(2~3~4\right) \\
        & = \tau_2 \circ \left(1~2\right) \circ \left(2~4\right) \\
        & = \left(2~3\right) \circ \left(1~2~4\right) \\
        & = \left(1~3~2~4\right) = q \\
        \tau_2 \circ \tau_3 \circ \tau_2 \circ \tau_1 & = \tau_2 \circ \tau_3 \circ \left(2~3\right) \circ \left(1~2\right) \\
        & = \tau_2 \circ \left(3~4\right) \circ \left(1~3~2\right) \\
        & = \left(2~3\right) \circ \left(1~4~3~2\right) \\
        & = \left(1~4~2\right) = r
    \end{split}
\end{equation} 
\subsection*{Part (b)}
The symmetric group $S_4$ is generated by the set of (adjacent) transpositions $\left\{\tau_1, \tau_2, \tau_3\right\}$. Letting $p := \left(1~2~3~4\right), s := \left(1~2\right)$. $\left\{\tau_1, \tau_2, \tau_3\right\}$ can be generated by $p, s$ as follows:
\begin{equation}
    \begin{split}
        s & = \left(1~2\right) = \tau_1 \\
        p \circ p \circ s \circ p \circ s & = p \circ p \circ s \circ \left(1~2~3~4\right) \circ \left(1~2\right) \\
        & = p \circ p \circ \left(1~2\right) \circ \left(1~3~4\right) \\
        & = p \circ \left(1~2~3~4\right) \circ \left(1~3~4~2\right) \\
        & = \left(1~2~3~4\right) \circ \left(1~4~3\right) \\
        & = \left(2~3\right) = \tau_2 \\
        p \circ p \circ s \circ p \circ p & = p \circ p \circ s \circ \left(1~2~3~4\right) \circ \left(1~2~3~4\right) \\
        & = p \circ p \circ \left(1~2\right) \circ \left(1~3\right)\left(2~4\right) \\
        & = p \circ \left(1~2~3~4\right) \circ \left(1~3~2~4\right) \\
        & = \left(1~2~3~4\right) \circ \left(1~4~2\right) \\
        & = \left(3~4\right) = \tau_3
    \end{split}
\end{equation}
Evidently, $\left\{\tau_1, \tau_2, \tau_3\right\}$ is generated by $\left\{\left(1~2\right), \left(1~2~3~4\right)\right\}$. Since $S_4$ is generated by $\left\{\tau_1, \tau_2, \tau_3\right\}$, $S_4$ is generated by $\left\{\left(1~2\right), \left(1~2~3~4\right)\right\}$.

\clearpage

\problem
\subsection*{Part (a)}
For some $n \times n$ permutation matrix $P$, the product $PP^T$ is defined such that:
\begin{equation}
    \begin{split}
        \left(PP^T\right)_{ij} & = \sum_{k=1}^n P_{ik}P^T_{kj} \\
        & = \sum_{k=1}^n P_{ik}P_{jk}
    \end{split}
\end{equation}
Note that permutation matrices, by definition, have a single $1$ in each row and in each column, and $0$ for the remaining entries. The above expression then simplifies to:
\begin{equation}
    \begin{split}
        \left(PP^T\right)_{ij} & = \delta_{ij}
    \end{split}
\end{equation}
As every entry in the product is then equal to the corresponding entry in the identity matrix, $PP^T = I$. Therefore $P^{-1} = P^T$, and the transpose of a permutation matrix is its inverse.
\subsection*{Part (b)}
For some $n \times n$ permutation matrix $P$, $\det\left(P^T\right) = \det\left(P\right)$.
Then:
\begin{equation}
    \begin{split}
        PP^{-1} = PP^T & = I \\
        \det\left(PP^T\right) & = \det\left(I\right) \\
        \det\left(P\right) \cdot \det\left(P^T\right) & = 1 \\
        \det\left(P\right) \cdot \det\left(P\right) = \det\left(P\right)^2 & = 1 \\
        1 - \det\left(P\right)^2 = \left(1 + \det\left(P\right)\right)\left(1 - \det\left(P\right)\right) = 0 \\
        \det\left(P\right) & = \pm 1
    \end{split}
\end{equation}
Therefore the determinant of a permutation matrix is always $\pm 1$.
\subsection*{Part (c)}
The identity matrix, having determinant $1$, is an even permutation.  We can then express $p$ as:
\begin{equation}
    \begin{split}
        p = \tau_{i_1} \circ \tau_{i_2} \circ \ldots \circ \tau_{i_k} \circ I
    \end{split}
\end{equation}
Since $\tau_{i_k} \circ I$ is obtained from $I$ by interchanging two different rows one time, $\det\left(\tau_{i_k} \circ I\right) = -\det\left(I\right) = -1$. This can be extended until $p$, which is obtained from $I$ by interchanging two different rows $k$ times, which means that $\det\left(p\right) = \left(-1\right)^k\det\left(I\right) = \left(-1\right)^k$. Evidently, $\sgn\left(p\right) = \det\left(p\right) = 1$ when $k$ is even, and $\sgn\left(p\right) = \det\left(p\right) = -1$ when $k$ is odd. Therefore $p$ is even if $k$ is even, and $p$ is odd if $k$ is odd.

Let there be some even $p$ and suppose that $k$ is odd. Since $p = \tau_{i_1} \circ \tau_{i_2} \circ \ldots \circ \tau_{i_k} \circ I$ is obtained from $I$ by interchanging two different rows $k$ times, $\det\left(p\right) = \left(-1\right)^k\det\left(I\right) = \left(-1\right)^k$.  Since $k$ is odd, $\sgn\left(p\right) = \det\left(p\right) = -1$. However, this is a contradiction; as $p$ is even, $\sgn\left(p\right) = 1$. Consequently $k$ must be even. Now let there be some odd $p$ and suppose that $k$ is even.Since $p = \tau_{i_1} \circ \tau_{i_2} \circ \ldots \circ \tau_{i_k} \circ I$ is obtained from $I$ by interchanging two different rows $k$ times, $\det\left(p\right) = \left(-1\right)^k\det\left(I\right) = \left(-1\right)^k$. Since $k$ is even, $\sgn\left(p\right) = \det\left(p\right) = 1$. However, this is a contradiction; as $p$ is odd, $\sgn\left(p\right) = -1$. Consequently $k$ must be odd.

Therefore $k$ is even if $p$ is even, and $k$ is odd if $p$ is odd, which means that $p$ is even iff $k$ is even, and $p$ is odd iff $k$ is odd.
\end{document}